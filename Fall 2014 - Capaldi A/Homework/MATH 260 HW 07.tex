\documentclass{article}
\usepackage{amsmath,amssymb,amsfonts}
\usepackage{graphicx}
\usepackage{multicol}
\setlength{\oddsidemargin}{0in} \setlength{\evensidemargin}{0in}
\setlength{\topmargin}{0in} \setlength{\textheight}{8.5in}
\setlength{\textwidth}{6.5in}
\pagestyle{empty}

\begin{document}
\begin{flushleft}
	\bfseries{MATH 260, Homework 7, Fall `14}\\
	\bfseries{Due: October 10, 2014 at 2:20 PM}\\
	\bfseries{Honor Code:} \hspace{3.5in}\bfseries{Name:}\\
	\hspace{4.37in}\bfseries{Section:}
\end{flushleft}
\begin{flushleft}
\vspace{.25in}

1) (25 pts) Decide whether or not the given set constitutes a subspace of $\mathbb{R}^n$.  If it's not a subspace, identify at least one requirement that is not satisfied and demonstrate it with an example of it failing.

\vspace{0.2in}

a) $S_1 = \{(x,y)| y = 2\}$

\vspace{2.5in}

b) $S_2 = \{(x,y,z)| x + y = 0\}$

\vspace{2.5in}

c) $S_3 = \{(x,y,z)| x + z = 3\}$

\pagebreak

d) $S_4 = \{(x,y)| x^2 + y^2 = 1\}$

\vspace{2.5in}

e) $S_5$ is the set of all points in the 1st quadrant of the Cartesian plane.

%2) (12 pts) Determine if the following sets of ``vectors" are linearly independent.  Briefly explain your reasoning.
%
%\vspace{0.2in}
%
%a) $\left\{ \begin{bmatrix} 2\\4\\6\\ \end{bmatrix}, \begin{bmatrix} 1\\2\\3\\ \end{bmatrix} \right\}$
%
%\vspace{1.5in}
%
%b) $\left\{ \begin{bmatrix} 1\\1\\1\\ \end{bmatrix}, \begin{bmatrix} 0\\1\\1\\ \end{bmatrix}, \begin{bmatrix} 0\\0\\-1\\\end{bmatrix} \right\}$
%
%
%\vspace{1.5in}
%
%c) $\{2t^2 + 1, t, 3t + 4, -t^2\}$
%
%\vspace{1.5in}
%
%d) $\left\{\begin{bmatrix} 1&0\\0&1\\1&0\\ \end{bmatrix}, \begin{bmatrix} 0&1\\1&0\\0&1\\ \end{bmatrix}, \begin{bmatrix} 1&1\\1&1\\1&1\\ \end{bmatrix} \right\}$
%
%\pagebreak
%
%3) (16 pts) For each vector space below, determine if the accompanying set is a basis for it.  If it is not, add or remove vectors from the set until your new set is a basis for the vector space.  State the dimension of the vector space.
%
%\vspace{0.2in}
%
%a) $\mathbb{R}^2$; \hspace{0.1in} $\left\{\begin{bmatrix} 1\\2\\\end{bmatrix}, \begin{bmatrix} 3\\4\\\end{bmatrix}, \begin{bmatrix} 5\\6\\\end{bmatrix} \right\}$
%
%\vspace{2in}
%
%b) $\left\{ \begin{bmatrix} x\\y\\z\\\end{bmatrix} | x+y=0 \right\}$; \hspace{0.1in} $\left\{ \begin{bmatrix} -1\\1\\0\\\end{bmatrix} \right\}$
%
%\vspace{2in}
%
%c) the set of diagonal $3 \times 3$ matrices; \hspace{0.1in} $\left\{ \begin{bmatrix} 1&0&0\\0&1&0\\0&0&1\\ \end{bmatrix} \right\}$
%
%\vspace{2in}
%
%d) the set of all upper triangular $2 \times 2$ matrices; \hspace{0.1in} $\left\{ \begin{bmatrix} 1&0\\0&1\\ \end{bmatrix}, \begin{bmatrix} 0&1\\0&1\\ \end{bmatrix}, \begin{bmatrix} 1&1\\0&0\\ \end{bmatrix} \right\}$
%
%\pagebreak
%
%4) (12 pts) Find the null space and the column space for the matrix $\textbf{A} = \begin{bmatrix} 1&2&3&4 \\ 2&3&4&5\\ 3&5&7&9 \end{bmatrix}$.  Give the dimension of each of the two subspaces you found.
%
%\pagebreak
%
%5) (12 pts) Re-do problem 4 using $\textbf{A}^T$ instead.
%
%\vspace{6.5in}
%
%6) (6 pts) Find the sum of the dimensions of the two subspaces from problem 4, then do so again for problem 5.  Note that $\textbf{A}$ is $3 \times 4$.  How do the dimensions of these subspaces relate to the dimensions of $\textbf{A}$?


\end{flushleft}
\end{document}
