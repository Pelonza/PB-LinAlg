\documentclass{article}
\usepackage{amsmath,amssymb,amsfonts}
\usepackage{graphicx}
\usepackage{multicol}
\setlength{\oddsidemargin}{0in} \setlength{\evensidemargin}{0in}
\setlength{\topmargin}{0in} \setlength{\textheight}{8.5in}
\setlength{\textwidth}{6.5in}
\pagestyle{empty}

\begin{document}
\begin{flushleft}
	\bfseries{MATH 260, Homework 3, Fall `14}\\
	\bfseries{Due: September 12, 2014 at 2:20 PM}\\
	\bfseries{Honor Code:} \hspace{3.5in}\bfseries{Name:}\\
	\hspace{4.37in}\bfseries{Section:}
\end{flushleft}
\begin{flushleft}
\vspace{.25in}

1) Consider the following linear system.

\begin{equation*}
\begin{array}{ccccccr}
2x &   &    & + & 6z & = & -10\\
2x &   &    & + & 2z & = & -2\\
   &   &  y & - & 8z & = & 19\\

\end{array}
\end{equation*}

a) (3 pts) Write the system in $\textbf{A}\vec{\textbf{x}} = \vec{\textbf{b}}$ form.

\vspace{1in}

b) (3 pts) Write the augmented matrix for the system.

\vspace{1in}

c) (12 pts) Use elementary row operations to convert the augmented matrix for the system into REF.  Specify the operations used at each step.

\pagebreak

d) (3 pts) Write the linear system that corresponds to the REF augmented matrix you found in part (c).

\vspace{1.5in}

e) (4 pts) Solve the linear system from part (d) using ``back" substitution.  (Start with the bottom equation to get $z$, substitute into the previous equation to get $y$, then substitute those into the first equation to get $x$).

\end{flushleft}

\end{document}
