\documentclass{article}
\pagestyle{empty}
\usepackage{amsmath,amssymb,amsfonts}
\usepackage{graphicx}
\usepackage{multicol}
\setlength{\oddsidemargin}{0in} \setlength{\evensidemargin}{0in}
\setlength{\topmargin}{0in} \setlength{\textheight}{8.5in}
\setlength{\textwidth}{6.5in}

\begin{document}
\begin{flushleft}
	\bfseries{MATH 260, Linear Systems and Matrices, Fall `14}\\
	\bfseries{Final Study Goals}\\
\end{flushleft}
\begin{flushleft}
This is a list of learning goals you should be able to demonstrate that you have achieved on the final.  This is not a complete list of all topics the final could cover.  All topics for the final will be pulled from the reading (sections 3.1-3.6, 5.3-5.4 of the book), the videos, the homework and  activities.  The major focus of the questions will be material from Weeks 7-12 (after midterm), but you still must know the material from Weeks 1-6 as the later material relies on it.

\vspace{0.2in}

\begin{enumerate}
\item Be able to identify a vector space
\item Be able to calculate linear combinations and multiples of vectors
\item Be able to verify or disprove a set as a subspace by using The Subspace Theorem
\item Be able to give non-trivial subspaces of common vector spaces
\item Be able to determine if a set of vectors is linearly independent
\item Be able to explain the span of a set, the relationship between span and the basis of a vector space
\item Be able to determine if a set of vectors is a basis for a vector space
\item Be able to find a basis for a subspace
\item Be able to find the dimension of a subspace
\item Be able to find the null space of a matrix
\item Be able to give the column space for a matrix
\item Be able to give a basis for the null space and for the column space of a matrix
\item Be able to describe how the null space and column space relate to each other
\item Be able to use the Rank-Nullity theorem
\item Be able to calculate the eigenvalues and eigenvectors of a matrix
\item Be able to identify several properites of eigenvalues/vectors
\item Be able find eigenspaces
\item Be able to find eigenvectors for repeated eigenvalues
\item Understand how repeated eigenvalues and their eigenvectors affect dimension of subspaces
\item Be able diagonalize a matrix
\item Be able to identify matrices that cannot be diagonalized
\item Be able to use the diagonalization of a matrix to find a matrix power
\end{enumerate}

\pagebreak

Old goals from midterm:

\begin{enumerate}
\item Be able to perform basic matrix-scalar operations (such as matrix addition, matrix multiplication, etc.).
\item Be able to transpose a matrix.
\item Be able to convert from a system of equations to a (augmented) matrix, and back.
\item Be able to perform elementary row operations on a matrix.
\item Be able to identify row echelon form (REF), and be able to manipulate a matrix to be in REF.
\item Be able to identify reduced row echelon form (RREF) and manipulate a matrix to be in RREF.
\item Be able to state the number of solutions (and find them) to a system, based on its RREF.
\item Be able to tell what the rank of a matrix is.
\item Know what the implications of a square matrix not having full rank are.
\item Be able to find the inverse of a matrix.
\item Be able to identify several characteristics of an invertible matrix. 
\item Be able to state the number of solutions to a matrix-vector equation, based on invertibility.
\item Be able to determine when a matrix has no inverse.
\item Be able to find the determinant of a matrix (up to $4 \times 4$, but understand how it would work beyond).
\item Be able to find the `minor' and `cofactor' of elements in a matrix. 
\item Discover several properties of determinants including if a matrix has an inverse.
\end{enumerate}

\end{flushleft}
\end{document}