\documentclass{article}
\pagestyle{empty}
\usepackage{amsmath,amssymb,amsfonts}
\usepackage{graphicx}
\usepackage{multicol}
\setlength{\oddsidemargin}{0in} \setlength{\evensidemargin}{0in}
\setlength{\topmargin}{0in} \setlength{\textheight}{8.5in}
\setlength{\textwidth}{6.5in}

\begin{document}
\begin{flushleft}
	\bfseries{MATH 260, Linear Systems and Matrices, Fall `14}\\
	\bfseries{Midterm Study Goals}\\
\end{flushleft}
\begin{flushleft}
This is a list of learning goals you should be able to demonstrate that you have achieved on the midterm.  This is not necessarily a complete list of all topics the midterm could cover, but it's close.  All topics for the midterm will be pulled from the reading (sections 3.1-3.4 of the book), the videos, the homework and activities.

\vspace{0.2in}

\begin{enumerate}
\item Be able to perform basic matrix-scalar operations (such as matrix addition, matrix multiplication, etc.).
\item Be able to transpose a matrix.
\item Be able to convert from a system of equations to a (augmented) matrix, and back.
\item Be able to perform elementary row operations on a matrix.
\item Be able to identify row echelon form (REF), and be able to manipulate a matrix to be in REF.
\item Be able to identify reduced row echelon form (RREF) and manipulate a matrix to be in RREF.
\item Be able to state the number of solutions (and find them) to a system, based on its RREF.
\item Be able to tell what the rank of a matrix is.
\item Know what the implications of a square matrix not having full rank are.
\item Be able to find the inverse of a matrix.
\item Be able to identify several characteristics of an invertible matrix. 
\item Be able to state the number of solutions to a matrix-vector equation, based on invertibility.
\item Be able to determine when a matrix has no inverse.
\item Be able to find the determinant of a matrix (up to $4 \times 4$, but understand how it would work beyond).
\item Be able to find the `minor' and `cofactor' of elements in a matrix. 
\item Discover several properties of determinants including if a matrix has an inverse.
\end{enumerate}

\end{flushleft}
\end{document}