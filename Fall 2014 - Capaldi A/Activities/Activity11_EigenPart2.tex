\documentclass{article}
\pagestyle{empty}
\usepackage{amsmath,amssymb,amsfonts}
\usepackage{graphicx}
\usepackage{multicol}
\setlength{\oddsidemargin}{0in} \setlength{\evensidemargin}{0in}
\setlength{\topmargin}{0in} \setlength{\textheight}{8.5in}
\setlength{\textwidth}{6.5in}

\makeatletter
\renewcommand*\env@matrix[1][*\c@MaxMatrixCols c]{%
	\hskip -\arraycolsep
	\let\@ifnextchar\new@ifnextchar
	\array{#1}}
\makeatother

\begin{document}
\begin{flushleft}
	\bfseries{MATH 260, Linear Systems and Matrices, Fall `14}\\
	\bfseries{Activity 11:  Eigenvalues and Eigenvectors, Part 2}\\
	%\bfseries{Honor Code:} \hspace{3.5in}\bfseries{Names:}\\\end{flushleft}
\begin{flushleft}

\section*{Warm-up:  Subspaces}
Recall from Lesson 7...

\vspace{0.1in}

\textbf{Vector Subspace Theorem:} a nonempty subset $W$ of a vector space $V$ is a \textbf{subspace} of $V$ if it is closed under addition and scalar multiplication:

\vspace{0.2in}

(i) If $\vec{x},\vec{y}\in W$, then $\vec{x}+\vec{y}\in W$ \hspace{0.5in} (ii) If $\vec{x}\in W$ and $c\in \mathbb{R}$, then $c\vec{x} \in W$

\vspace{0.2in}

1) Sketch the lines: $y=x$ and $ y=x+1$.  One line is a subspace of $\mathbb{R}^2$, one is not. Decide, then explain your reasoning.

\vspace{1.5in}

\newpage

\section*{Eigenspaces}

2)a) In Activity 10 you had:
$\textbf{A}=
\begin{bmatrix} 
1 & 1 \\ 4&1 
\end{bmatrix}$
 which had eigenvectors:
$\vec{v}_1=\begin{bmatrix} 1 \\ 2 \end{bmatrix}$,
and $\vec{v}_2=\begin{bmatrix} 1 \\ -2 \end{bmatrix}$. \\
\hspace{0.12in} You also found that $\textbf{A}\vec{v}_1 = \begin{bmatrix} 3 \\ 6 \end{bmatrix}$. Sketch $\textbf{A}\vec{v}_1$ and $\vec{v}_1$ (on the same graph).

\vspace{2in}

b) Looking back at (1), what can you say about $span\{ \vec{v}_1 \}$? 

\vspace{1in}

c) Sketch $span\{ \vec{v}_2 \}$. Is $span\{ \vec{v}_2 \}$ a subspace of $\mathbb{R}^2$ ?   

\vspace{2in}

We call the $span\{ \vec{v}_i \}$ the \textbf{eigenspace} $\mathbb{E}_{\lambda}$. \\

\vspace{0.2in}

3) What is the dimension of each eigenspace?

\newpage

\section*{Repeated Eigenvalues}

4) What are the eigenvalues and eigenvectors of: 
$\textbf{H}=
\begin{bmatrix}
3 & 0 \\
5 & 3
\end{bmatrix}$

\vspace{3in}

Repeated eigenvalues sometimes have more than one eigenvector though.

\vspace{0.1in}

5) The eigenvalues for this matrix are $\lambda_{1,2}=1$ and $\lambda_3 = 2$. Find the eigenvectors associated with these eigenvalues for the matrix: 
$\textbf{R}=
\begin{bmatrix}
0 & 0 & 2 \\
-1 & 1 & 2 \\
-1 & 0 & 3
\end{bmatrix} $\\


\newpage

We know that eigenvectors always satisfy the 2nd property of vector subspaces (you should be able to explain why!). We also found above that the eigenspace from a single eigenvector is a valid subspace. However, notice that we wrote our eigenspace as $\mathbb{E}_{\lambda}$. But there are TWO eigenvectors with our double eigenvalue. 

\vspace{0.2in}

Let's denote the double eigenvalue, and its eigenvectors $\lambda_1 , \lambda_2$ and $\vec{v}_1 , \vec{v}_2$ respectively. 

\vspace{0.2in}

%6) Show that a linear combination of $\vec{v}_1$ and $\vec{v}_2$ (e.g. $\vec{v}_1 + \vec{v}_2$ ) is STILL an eigenvector for $\lambda_{1,2}$. In other words it satisfies the eigenvalue equation:\\
%\begin{equation*}
%\textbf{R}(\vec{v}_1 + \vec{v}_2 ) = \lambda_1 (\vec{v}_1 + \vec{v}_2 )
%\end{equation*}

%\vspace{1.75in}

6) State all the eigenspaces and their corresponding dimension for \textbf{R}?

\vspace{2in}

7) Give a geometric interpretation of $span\{ \vec{v}_1 ,  \vec{v}_2 \}$ in $\mathbb{R}^3$ (3-d space). \\
\textit{Hint: this is a special shape in $\mathbb{R}^3$}

\vspace{1in}

The algebraic multiplicity of an eigenvalue is the number of times it is a root (e.g. $a$ in $(\lambda -\lambda_i)^a$)\\
The geometric multiplicity of an eigenvalue is the dimension of its corresponding eigenspace.\\
8)a) Give the algebraic and geometric multiplicity for the eigenvalues of \textbf{R}.

\vspace{1in}

b) Give the algebraic and geometric multiplicity for the eigenvalues of \textbf{H}.

\vspace{1in}

c) How do the algebraic and geometric multiplicity compare to each other?

\newpage

\section*{Distinct Eigenvalues}

There is very useful theorem about distinct eigenvalues:\\
\textbf{Distinct Eigenvalue Theorem:}
Let \textbf{A} be an $n \times n$ matrix. If $\lambda_1 , \lambda_2, \ldots , \lambda_p$ are distinct eigenvalues with corresponding eigenvectors $\vec{v}_1,\vec{v}_2, \ldots, \vec{v}_p$ then $\{ \vec{v}_1,\vec{v}_2, \ldots, \vec{v}_p \}$ is a set of linearly independent vectors. \\

\vspace{0.1in}

9)a) Find the eigenvalues and an eigenvectors for each value for: 
$\textbf{W}=
\begin{bmatrix}
3 & 0 \\
5 & 2 
\end{bmatrix}$

\vspace{2in}

b) Show that the eigenvectors are linearly independent.

\vspace{1.5in}

10) There's something to be careful of with this theorem though...You found the eigenvectors for \textbf{R} in (3). Are the eigenvectors of \textbf{R} linearly independent?

\vspace{1.5in}

11) If the eigenvectors are linearly independent are you guaranteed to have distinct eigenvalues? 


\end{flushleft}
\end{document}