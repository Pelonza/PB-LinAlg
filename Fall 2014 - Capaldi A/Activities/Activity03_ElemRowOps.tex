\documentclass{article}
\pagestyle{empty}
\usepackage{amsmath,amssymb,amsfonts,soul}
\usepackage{graphicx}
\usepackage{multicol}
\setlength{\oddsidemargin}{0in} \setlength{\evensidemargin}{0in}
\setlength{\topmargin}{0in} \setlength{\textheight}{8.5in}
\setlength{\textwidth}{6.5in}

\makeatletter
\renewcommand*\env@matrix[1][*\c@MaxMatrixCols c]{%
	\hskip -\arraycolsep
	\let\@ifnextchar\new@ifnextchar
	\array{#1}}
\makeatother

\begin{document}
\begin{flushleft}
	\bfseries{MATH 260, Linear Systems and Matrices, Fall `14}\\
	\bfseries{Activity 3:  Elementary Row Operations}\\
%	\bfseries{Honor Code:} \hspace{3.5in}\bfseries{Names:}\\
\end{flushleft}
\begin{flushleft}

\section*{Algebra Warmup}
Solve the following system of equations by elimination. Everytime you add two equations, or multiply by numbers, right out explicitly what you are adding together or multiplying by on the left.

\vspace{0.25in}

\begin{center}
$\begin{array}{rrrrrcr}
x&+&y&+&z&=&3\\
2x&-&2y&-&z&=&-9\\
-x&+&y&-&z&=&3
\end{array}
$
\end{center}

\newpage

\section{Turning Systems into Matrices}

Linear Algebra is a mathematical field which allows fast and easy solving of systems of linear equations by denoting operations on the entire system succinctly.  Notice that it was rather messy to work with all of those equations in the warmup excercise; imagine doing that for 5 or 10 variables and equations. There are two notations we use in Linear Algebra to condense systems: \textit{Matrix-Vector Form} and \textit{Augmented Matrices}. Let's turn the system from the warmup into Matrix-Vector Form:

\vspace{0.2in}

a) First, we'll identify all the variables of the system and put them together into one column vector of size $3 \times 1$: $\vec{\textbf{x}}=\begin{bmatrix} x\\y\\z\end{bmatrix}$

\vspace{0.2in}

b) Make a matrix of the variables' coefficients; we'll call it \textbf{A}.  It should have size $3 \times 3$.  Write it out. What would $\textbf{A}\cdot\vec{\textbf{x}}$ be?

\vspace{1.5in}

c) Write the values on the right-hand sides of the $=$ as a $3 \times 1$ column vector.  Call it $\vec{\textbf{b}}$.

\vspace{1in}

d) Using the vectors $\vec{\textbf{x}}$, $\vec{\textbf{b}}$ and the matrix \textbf{A}, express the system of equations from the warmup as a single matrix-vector equation.

\vspace{1.5in}

e) Augmented Matrix form is where we just the coefficients (\textbf{A}) and the right-hand-side ($\vec{\textbf{b}}$). We write is as: $[\textbf{A}|\vec{\textbf{b}}]$. Generally, we actually write out the numbers, as we want to manipulate the augemented matrix.  Write out the full augmented matrix for this system (it should be a $3 \times 4$ matrix with a vertical line before the last column).

\newpage

\section{Row Operations}

The main reason we want an augemented matrix is to use it for solving systems. One way to solve the system is by doing row operations to get it into \textit{row echelon form} (REF) or \textit{reduced row echelon form} (RREF). We'll define these in a bit, once we are comfortable with operations...\\

\vspace{0.1in}

\hrulefill \\
\noindent
There are three basic row operations we can perform (see page 134 of the text):\\
$R_i$ means row \textit{i} before an operation while $R_i^*$ denotes row \textit{i} after an operation.\\
1) Row swap between row \textit{i} and row \textit{j}, denoted: $R_i \leftrightarrow R_j$\\
2) Multiply a row \textit{i} by a constant \textit{c} ( with $c\neq 0$) , denoted: $R_i^*=c R_i$\\
3) Add a (multiple) of a row to another row, denoted: $R_i^*= R_i+c R_j$\\
\hrulefill \\

\vspace{0.1in}

\noindent

Lets try these out with our augmented matrix version of the warmup problem:

\vspace{0.2in}

a) Perform the following row operations in sequence:\\

\vspace{0.1in}

$\begin{array}{c}
\\
\\
R_2 \leftrightarrow R_3\\
\\
\\
\\
\end{array}
$
\hspace{0.55in}
$\begin{bmatrix}[ccc|c]
\hspace{0.2in}& \hspace{0.2in}& \hspace{0.2in}& \hspace{0.2in}\\
& & & \\
\hspace{0.2in}& \hspace{0.2in}& \hspace{0.2in}& \hspace{0.2in}\\
& & & \\
\hspace{0.2in}& \hspace{0.2in}& \hspace{0.2in}& \hspace{0.2in}\\
& & & \\
\end{bmatrix}$
\hspace{0.2in}
$\begin{array}{c}
\\
\\
\\
\\
R_3^*=R_3+2 R_2\\
\end{array}
$
\hspace{0.25in}
$\begin{bmatrix}[ccc|c]
\hspace{0.2in}& \hspace{0.2in}& \hspace{0.2in}& \hspace{0.2in}\\
& & & \\
\hspace{0.2in}& \hspace{0.2in}& \hspace{0.2in}& \hspace{0.2in}\\
& & & \\
\hspace{0.2in}& \hspace{0.2in}& \hspace{0.2in}& \hspace{0.2in}\\
& & & \\
\end{bmatrix}$\\
\vspace{0.1in}
$\begin{array}{c}
\\
\\
R_2^* = R_1 + R_2\\
\\
\end{array}
$
\hspace{0.25in}
$\begin{bmatrix}[ccc|c]
\hspace{0.2in}& \hspace{0.2in}& \hspace{0.2in}& \hspace{0.2in}\\
& & & \\
\hspace{0.2in}& \hspace{0.2in}& \hspace{0.2in}& \hspace{0.2in}\\
& & & \\
\hspace{0.2in}& \hspace{0.2in}& \hspace{0.2in}& \hspace{0.2in}\\
& & & \\
\end{bmatrix}$
\hspace{0.2in}
$\begin{array}{c}
\\
\\
R_2^*= \frac{1}{2}R_2\\
\\
R_3^*= -\frac{1}{3}R_3\\
\end{array}
$
\hspace{0.5in}
$\begin{bmatrix}[ccc|c]
\hspace{0.2in}& \hspace{0.2in}& \hspace{0.2in}& \hspace{0.2in}\\
& & & \\
\hspace{0.2in}& \hspace{0.2in}& \hspace{0.2in}& \hspace{0.2in}\\
& & & \\
\hspace{0.2in}& \hspace{0.2in}& \hspace{0.2in}& \hspace{0.2in}\\
& & & \\
\end{bmatrix}$\\
\vspace{0.1in}
You should now have an augmented matrix with 8 non-zero entries.  The first non-zero entry in each row should also be a positive 1.  Now, let's convert this augmented matrix back into equations...\\
\vspace{0.1in}
b) Rewrite your final augmented matrix below, without the brackets or line (but leave space between each number). Now, beside each number in column 1 put an `$x$', column 2 a `$y$' and column 3 a `$z$'. Where the vertical line was, put an `=' sign.

\vspace{1.25in}

c) How does this compare to what you got for the warmup?  How do the row operations compare to what you did to solve by elimination?

\vspace{1.5in}

\newpage
\section{REF and RREF}

The matrix we got at the end of 2a was in \textit{row echelon form}. A matrix in REF has the following traits (see page 136 of the text):

\begin{enumerate}
	\item Any zero rows are at the bottom
	\item The leftmost non-zero entry of each non-zero row equals 1.\\
 (This entry is called its \textbf{pivot} or \textbf{leading 1}.)
	\item Each pivot is further to the right than the pivot in the row above it.
\end{enumerate}

a) Determine if each of the following matrices is in REF or not. If not, perform row operations to turn it into REF.\\

\vspace{0.1in}
\begin{center}
$\textbf{M}=\begin{bmatrix}
0 & 0 & 2\\
1 & 3 & 1
\end{bmatrix}
$
\hspace{0.2in}
$\textbf{T}=\begin{bmatrix}
1 & 3 & 0 & -1 \\
0 & 0 & 1 & -2
\end{bmatrix}$\\
\end{center}

\vspace{2.5in}

\textit{Reduced row echelon form} (RREF) is a even more strict form for matrices to take, which adds an additional property to the three above:\\
\vspace{0.1in}
\hrulefill \\
\indent 4. Each pivot is the only non-zero entry in its column.\\
\vspace{-3pt}
\hrulefill \\
\vspace{0.1in}
Matrix \textbf{T} from (a) is actually in RREF.\\


b) What additional row operation(s) are required to turn matrix \textbf{M} into RREF?

\newpage

\section{Bringing it Together}

Take the following system of equations, write it first in matrix-vector from, then as an augmented matrix. Finally, use row operations to turn it into RREF:

\begin{center}
$\begin{array}{rrrrrrr}
x_1 & + & x_2 & + & 2 x_3 & = & 1\\
2 x_1 & - & x_2 & + & x_3 & = & 2\\
4 x_1 & + & x_2 & + & 5 x_3 & = & 4
\end{array}$
\end{center}

\end{flushleft}
\end{document}