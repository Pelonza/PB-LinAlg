\documentclass{article}
\pagestyle{empty}
\usepackage{amsmath,amssymb,amsfonts}
\usepackage{graphicx}
\usepackage{multicol}
\setlength{\oddsidemargin}{0in} \setlength{\evensidemargin}{0in}
\setlength{\topmargin}{0in} \setlength{\textheight}{8.5in}
\setlength{\textwidth}{6.5in}

\begin{document}
\begin{flushleft}
	\bfseries{MATH 260, Linear Systems and Matrices, Fall `14}\\
	\bfseries{Activity 2:  Matrix Operations}\\
%	\bfseries{Honor Code:} \hspace{3.5in}\bfseries{Names:}\\
\end{flushleft}
\begin{flushleft}
\vspace{.25in}

Matrices for today's problems:\\

$\textbf{A}=\left[
\begin{array}{c}
	2\\
	4\\
	-1\\
\end{array}
\right]
$
\hspace{.1in}
$\textbf{B}=\left[
\begin{array}{ccc}
3 & 2 & -5
\end{array}\right]$
\hspace{.1in}

$\textbf{C}=\left[
\begin{array}{cc}
1 & 4\\
-3 & 5
\end{array}
\right]
$
\hspace{.1in}
$\textbf{D}=\left[
\begin{array}{cc}
2 & 1\\
4 & 3
\end{array}
\right]$
\hspace{.1in}
$\textbf{E}=\left[
\begin{array}{ccc}
3 & 8 & 2\\
-1 & x & x^2
\end{array}
\right]
$

\section*{Warmup: Constants and Matrices}

Find $k\textbf{C}$ where $k$ is a constant. What is $k\textbf{C}$ if $k=3$?

\vspace{2in}

\section*{Problem 1:  Vectors, and Vectors with Matrices}

a) Find the expected dimension of \textbf{AB}, \textbf{BI$_3$}, and \textbf{BE}.

\newpage

b) Now calculate each product.

\vspace{5in}

\section*{Problem 2: Matrix on Matrix}

a) Find \textbf{CD}.

\vspace{2in}

b) Make a prediction (don't calculate yet): Is the statement:  $\textbf{CD} = \textbf{DC}$ true or false?

\newpage

c) Find \textbf{DC}. Were you correct?

\vspace{1.5in}

d) Can you find any two matrices (\textbf{A} and \textbf{B}) for which $\textbf{AB}=\textbf{BA}$?  Note that here, \textbf{A} and \textbf{B} are NOT the matrices given on page 1. \textit{Hint: This is a SPECIAL circumstance.}

\vspace{1in}

\section*{Problem 3: More Matrix on Matrix}

a) Which of these are true for any generic matrices where the operations are well-defined (give some reasoning, you don't \textit{have} to prove them):\\
$(\textbf{AB})\textbf{C}=\textbf{A}+\textbf{BC}$ \hspace{0.2in} $\textbf{A}(\textbf{B}+\textbf{C})=\textbf{AB}+\textbf{AC}$ \hspace{0.2in} $(\textbf{B}+\textbf{C})\textbf{A}=\textbf{AB}+\textbf{CA}$

\vspace{2in}

b) Under certain circumstances, we can raise a matrix to a power.  Consider the $m \times n$ matrix $\textbf{A}$.  Discuss at your table what some requirements on $\textbf{A}$ might be, and what the resultant dimensions of $\textbf{A}^{k}$ would be.

\newpage

c) Find $\textbf{CC}$.  This is equivalent to $\textbf{C}^2$.

\vspace{1.5in}

d) Find $\textbf{C}^3$. \textit{Hint: Use your result from (c) to reduce your work}

\vspace{1.5in}

e) Could you find $\textbf{E}^3$?  If you can, give its dimensions.  If you can't find it, explain why not?

\vspace{0.75in}

\section*{Problem 4: Transposing Matrices}

We can find what's called the ``transpose" of a matrix by swapping the matrix's rows and columns.  More technically, to find the transpose of \textbf{A}, denoted by $\textbf{A}^\texttt{T}$ for all $i,j$ $[a_{ij}]^\texttt{T}=[a_{ji}]$.

\vspace{0.2in}

Find $\textbf{E}^\texttt{T}$.

\end{flushleft}
\end{document}