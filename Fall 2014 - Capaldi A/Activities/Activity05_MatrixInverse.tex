\documentclass{article}
\pagestyle{empty}
\usepackage{amsmath,amssymb,amsfonts}
\usepackage{graphicx}
\usepackage{multicol}
\setlength{\oddsidemargin}{0in} \setlength{\evensidemargin}{0in}
\setlength{\topmargin}{0in} \setlength{\textheight}{8.5in}
\setlength{\textwidth}{6.5in}

\makeatletter
\renewcommand*\env@matrix[1][*\c@MaxMatrixCols c]{%
	\hskip -\arraycolsep
	\let\@ifnextchar\new@ifnextchar
	\array{#1}}
\makeatother

\begin{document}
\begin{flushleft}
	\bfseries{MATH 260, Linear Systems and Matrices, Fall `14}\\
	\bfseries{Activity 5:  Matrix Inverses}\\
	%\bfseries{Honor Code:} \hspace{3.5in}\bfseries{Names:}\\
\end{flushleft}
\begin{flushleft}

\section*{Warmup: RREF Review}

Find the RREF of the following matrix:\\
\begin{center}
$\begin{bmatrix}[cc|cc]
1 & 1 &  1 & 0 \\
4 & 1 &  0 & 1 \\

\end{bmatrix}
$\\
\end{center}
\vspace{6in}

Congratulations! You've found your first matrix inverse.

\newpage

\section{The Identity Matrix and Inverses}

Notice that the right-hand side (RHS) of the warmup started as the identity matrix. The warmup exercise could be stated as: $[\textbf{A}|\textbf{I}]$. Reducing this combined matrix into RREF is the simplest way to find the inverse of a matrix. The inverse of the matrix $\textbf{A}$ is the RHS after getting the left-side into RREF.  That is, you start with $ [\textbf{A} | \textbf{I}] $ then row reduce to $ [\textbf{I} | \textbf{A}^{-1} ] $.

\vspace{0.2in}

a) One trait of the inverse is that the statement: $\textbf{A} \textbf{A}^{-1} = \textbf{I} $. Verify this is true for the warmup.

\vspace{1.5in}

b) Another trait if a matrix has an inverse, then its transpose has an inverse. Let's check that:

\vspace{0.2in}

(i) Find $\textbf{A}^T$. Recall that the transpose is defined as: $[a_{ij}]^T = [a_{ji}]$

\vspace{1.5in}

(ii) Find the inverse of $\textbf{A}^T$. \textit{Hint: start with the matrix  $\left[ A^T | I \right] $ }

\vspace{2in}

c) Another practice problem: Obtain the inverse of the following matrix using row operations and RREF.

\begin{center}
$\begin{bmatrix}[ccc]
1 & 1 & 1 \\
0 & 2 & 1 \\
1 & 0 & 1 
\end{bmatrix}$
\end{center}

\newpage

\section{A Use for the Inverse}

a) How did we use RREF to find solutions to the matrix-vector equation: $\textbf{A} \vec{\textbf{x}} = \vec{\textbf{b}}$?

\vspace{1in}

b) If we do the same operations to find a RREF, can't we use the inverse to find an answer too...?

\vspace{0.2in}

(i) Solve the following system by finding the RREF of the \textbf{\textit{augmented}} matrix:

\begin{center}
$\begin{array}{rrrrr}
x & + & y & = & -1\\
4x & + & y & = & -7
\end{array}
$ \\ \end{center}

\vspace{2in}

(ii) Now multiply the inverse you found in the warmup by $\vec{\textbf{b}}=\begin{bmatrix} -1 \\ -7 \end{bmatrix}$ 

\vspace{1.5in}

c) Consider your results and write down observations. Here's a little guidance...
Left multiply each side of the equation $\textbf{A}\vec{\textbf{x}}=\vec{\textbf{b}}$ by $\textbf{A}^{-1}$ and compare to answers you've obtained so far...

\vspace{2in}

\newpage

\section{More Solutions...}

a) Use the algorithm we discussed to find the inverse of the matrix:

\begin{center}
$\textbf{M} = 
\begin{bmatrix}[rrr]
3 & 0 & 3\\
-1 & 2 & 1\\
1 & 1 & 2
\end{bmatrix}
$\\\end{center}

\vspace{3in}

b) What happened? Based on your exploration in Section 2, if this matrix described a system, do you think it has a unique answer? Justify your answer.


\pagebreak

\section{Challenge}

For what value(s) $a, b$ and $c$ make each of the following matrices invertible.  If no such constant exists, say so and explain why.

\begin{center}
$\textbf{A}= \begin{bmatrix}
1 & 0 & a\\
0 & 1 & 0\\
0 & 0 & 1
\end{bmatrix} $
\hspace{1in}
$\textbf{B}= \begin{bmatrix}
1 & 0 & 1\\
0 & 1 & 0\\
b & b & b
\end{bmatrix} $
\hspace{1in}
$\textbf{C}= \begin{bmatrix}
1 & 1 & 0\\
0 & 0 & 1\\
1 & c & 0
\end{bmatrix} $
\end{center}

\end{flushleft}
\end{document}