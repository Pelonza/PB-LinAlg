\documentclass{article}
\pagestyle{empty}
\usepackage{amsmath,amssymb,amsfonts}
\usepackage{graphicx}
\usepackage{multicol}
\setlength{\oddsidemargin}{0in} \setlength{\evensidemargin}{0in}
\setlength{\topmargin}{0in} \setlength{\textheight}{8.5in}
\setlength{\textwidth}{6.5in}

\makeatletter
\renewcommand*\env@matrix[1][*\c@MaxMatrixCols c]{%
	\hskip -\arraycolsep
	\let\@ifnextchar\new@ifnextchar
	\array{#1}}
\makeatother

\begin{document}
\begin{flushleft}
	\bfseries{MATH 260, Linear Systems and Matrices, Fall `14}\\
	\bfseries{Activity 4:  Matrices, RREF, and Solutions to Systems}\\
	%\bfseries{Honor Code:} \hspace{3.5in}\bfseries{Names:}\\
\end{flushleft}
\begin{flushleft}

\section*{Row Operation Review}

\begin{center}
$\left[\begin{array}{rrr|r}
1 & 1 & 2 & 4\\
2 & -1 & 1 & -4\\
4 & 1 & 5 & 4\\
\end{array}\right]
$\\
\end{center}

There are three basic row operations we can perform (see page 134 of the text):\\
$R_i$ means row \textit{i} before an operation while $R_i^*$ denotes row \textit{i} after an operation.\\
1) Row swap between row \textit{i} and row \textit{j}, denoted: $R_i \leftrightarrow R_j$\\
2) Multiply a row \textit{i} by a constant \textit{c} ( with $c\neq 0$) , denoted: $R_i^*=c R_i$\\
3) Add a (multiple) of a row to another row, denoted: $R_i^*= R_i+c R_j$\\
\hrulefill \\

\vspace{0.2in}
\noindent
Lets try these out with our augmented matrix:\\
a) Perform the following row operations:\\
\vspace{0.2in}
$\begin{array}{c}
R_2^* = R_2 + (-2) R_1\\
\rightarrow \\
R_3^*=R_3+ (-4) R_1\\
\end{array}
$
\hspace{0.25in}
$\begin{bmatrix}[ccc|c]
\hspace{0.2in}& \hspace{0.2in}& \hspace{0.2in}& \hspace{0.2in}\\
& & & \\
\hspace{0.2in}& \hspace{0.2in}& \hspace{0.2in}& \hspace{0.2in}\\
& & & \\
\hspace{0.2in}& \hspace{0.2in}& \hspace{0.2in}& \hspace{0.2in}\\
& & & \\
\end{bmatrix}$
\hspace{0.2in}
$\begin{array}{c}
\\
\\
R_2^* = - \frac{1}{3} R_2 \\
\rightarrow \\
\\
\end{array}
$
\hspace{0.5in}
$\begin{bmatrix}[ccc|c]
\hspace{0.2in}& \hspace{0.2in}& \hspace{0.2in}& \hspace{0.2in}\\
& & & \\
\hspace{0.2in}& \hspace{0.2in}& \hspace{0.2in}& \hspace{0.2in}\\
& & & \\
\hspace{0.2in}& \hspace{0.2in}& \hspace{0.2in}& \hspace{0.2in}\\
& & & \\
\end{bmatrix}$\\
\vspace{0.1in}
$\begin{array}{c}
R_1^* = R_1 + (-1) R_2\\
\rightarrow \\
R_3^*=R_3+ (3) R_2\\
\end{array}$
\hspace{0.25in}
$\begin{bmatrix}[ccc|c]
\hspace{0.2in}& \hspace{0.2in}& \hspace{0.2in}& \hspace{0.2in}\\
& & & \\
\hspace{0.2in}& \hspace{0.2in}& \hspace{0.2in}& \hspace{0.2in}\\
& & & \\
\hspace{0.2in}& \hspace{0.2in}& \hspace{0.2in}& \hspace{0.2in}\\
& & & \\
\end{bmatrix}$\\

\vspace{0.2in}

You should now have a matrix with 5 non-zero entries.  The first non-zero entry in each row should be a positive 1.

\newpage
\section{RREF}
The matrix we got at the end of the warmup was in \textit{reduced row echelon form}. A matrix in RREF has the following traits (see page 136 of the text):
\begin{enumerate}
\item Any zero rows are at the bottom
\item The leftmost non-zero entry of each non-zero row equals 1.\\
 (This entry is called its \textbf{pivot} or \textbf{leading 1}.)
\item Each pivot is further to the right than the pivot in the row above it.
\item Each pivot is the only non-zero entry in its column.\\
\end{enumerate}
Determine if each of the following matrices is in RREF or not. If not, perform row operations to turn it into RREF.\\
\vspace{0.1in}
\begin{center}
$\textbf{M}=\begin{bmatrix}[ccc|c]
1 & 0 & -1 & 1\\
0 & 1 & 1 & 0\\
0 & 0 & 3 & -1
\end{bmatrix}$
\hspace{0.15in}
$\textbf{T}=\begin{bmatrix}[cc|c]
2 & -1 & 0\\
1 & -1 & -3
\end{bmatrix}$
\hspace{0.15in}
$\textbf{A}=\begin{bmatrix}[cccc|c]
1 & 0 & 0 & 2 & 5\\
0 & 1 & 0 & -2 & 2\\
0 & 0 & 1 & 5 & 6\\
0 & 0 & 0 & 0 & 2 \\
\end{bmatrix}$\\
\end{center}

\pagebreak

\section{Number of Solutions}

Two of the previous matrices, have the same traits in RREF, \textbf{M} and \textbf{T}.\\

\vspace{0.2in}

a) Describe the similarities between the two matrices in RREF.

\vspace{1.5in}

b) What's different about \textbf{A} and the matrix from the warmup?  \textit{Hint: Look at where zeros occur.}  Write out what the last line of matrix \textbf{A} represents in equation form.

\vspace{1.5in}

Just like when we dealt with systems of equations, augmented matrices (which are representing systems!) can have 3 types of solutions:  unique, infinitely many, or none.  For the first two (unique and infinitely many) the system is said to be `consistent'.  For the last type, no solutions, the system is `inconsistent'.

\vspace{0.2in}

c) Find which of the above matrices have unique solutions, infinite solutions, or no solutions by converting each back into equations (i.e. with $x_1$ ... $x_4$ as variables).

\newpage

\section{Rank and Pivots}

A \textbf{pivot column} of a matrix is any column that has a leading 1 in it (the rest of the entries in that column are zeros) once the matrix is put into RREF.  Note that if we're looking at augmented matrices, the last column (the one to the right of the vertical bar) is never a pivot column.

\vspace{0.2in}

Ex: For matrix \textbf{M}, all three columns are pivot columns.

\vspace{0.2in}

The \textbf{rank} of a matrix is the number of \textit{pivot columns} it has (again, once it is in RREF).

\vspace{0.2in}

Ex: For matrix \textbf{M} the rank is 3.

\vspace{0.2in}

a) Identify the pivot columns of each of \textbf{T} and \textbf{A} from page 2.

\vspace{2in}

b) Identify the rank of each of \textbf{T} and \textbf{A}.


\end{flushleft}
\end{document}