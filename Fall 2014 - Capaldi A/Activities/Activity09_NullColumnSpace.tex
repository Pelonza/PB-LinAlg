\documentclass{article}
\pagestyle{empty}
\usepackage{amsmath,amssymb,amsfonts}
\usepackage{graphicx}
\usepackage{multicol}
\setlength{\oddsidemargin}{0in} \setlength{\evensidemargin}{0in}
\setlength{\topmargin}{0in} \setlength{\textheight}{8.5in}
\setlength{\textwidth}{6.5in}

\makeatletter
\renewcommand*\env@matrix[1][*\c@MaxMatrixCols c]{%
	\hskip -\arraycolsep
	\let\@ifnextchar\new@ifnextchar
	\array{#1}}
\makeatother

\begin{document}
\begin{flushleft}
	\bfseries{MATH 260, Linear Systems and Matrices, Fall `14}\\
	\bfseries{Activity 9:  Null \& Column Spaces}\\
	%\bfseries{Honor Code:} \hspace{3.5in}\bfseries{Names:}\\
\end{flushleft}
\begin{flushleft}

\section*{Warm-up:  In the Null Space?}
%This is the matrix from Sect. 5.4, pg 339, #43
Recall that the null space is the set of all vectors $\vec{x}$ such that $\textbf{A}\vec{x}=\vec{0}$. Given the matrix:\\
\begin{center}
$\textbf{G}=
\begin{bmatrix}
2 & 3 & -1\\
-1 & 4 & 6 \\
1 &  7 & 5
\end{bmatrix}
$\\
\end{center}
1a) Determine which of the following vectors are in the null space of \textbf{G}.\\
\begin{center}
$ \vec{w}_1=\begin{bmatrix} 3 \\ 2 \\ 1 \end{bmatrix} $
\hspace{0.3in}
$ \vec{w}_2=\begin{bmatrix} 8 \\ -4 \\ 4 \end{bmatrix} $
\hspace{0.3in}
$ \vec{w}_3=\begin{bmatrix} -2 \\ 1 \\ -1 \end{bmatrix} $
\end{center}

\vspace{2in}

1b) Find the actual null space of \textbf{G}. Do your answers above make sense?

\newpage

\section*{Null Space}
Given the matrix $\textbf{F} = \begin{bmatrix} 1 & 2 & 1\\ 1 & 2 & 2 \end{bmatrix}$.

\vspace{0.1in}

2) Give a basis for the null space of $\textbf{F}$. 

\vspace{1.5in}

\section*{Column Space}
We know that the column space ( $span( \{ \vec{v}_1 , \vec{v}_2, \ldots \vec{v}_n \} )$ ) is a subspace. However, generally in mathematics we want to describe things in as simple terms as possible. For spaces, this means giving a \textit{basis} only for a space. \\
Recall that a set is a \textbf{basis} of a vector space if it has the properties:\\
(i) The set is linearly independent \hspace{.25in} (ii) The span of the set covers the entire vector space.\\

\vspace{0.2in}

3) The column space of \textbf{G} could be given as \textit{span}$ \left( \left\{ \begin{bmatrix} 2 \\ -1 \\ 1\end{bmatrix}, \begin{bmatrix} 3 \\ 4 \\ 7 \end{bmatrix}, \begin{bmatrix} -1 \\ 6 \\ 5 \end{bmatrix} \right\} \right)$. Does the set $S=\left\{ \begin{bmatrix} 2 \\ -1 \\ 1 \end{bmatrix}, \begin{bmatrix} 3 \\ 4 \\ 7 \end{bmatrix}, \begin{bmatrix} -1 \\ 6 \\ 5 \end{bmatrix} \right\}$ form a basis for the column space? (Show why or why not)

\vspace{1.5in}

We already have all the tools and information to define a valid basis. Let's do so...

\vspace{0.1in}

4) Which columns in the RREF of \textbf{G} are pivot columns?

\pagebreak

5) Take the columns from the original \textbf{G} that you identified as the pivot columns. Form a new set $S_{B}$ from these columns. These should be $3 \times 1$ vectors with non-zero values in each row.

\vspace{0.1in}

5a) Is this new set $S_{B}$ linearly independent?

\vspace{1.5in}

5b) Does it span the entire column space? (\textit{Hint: Does $span(S) = span (S_{B})$})

\vspace{1.5in}

5c) Is the set $S_{B}$ a basis for the column space? Explain.

\vspace{0.75in}

6) We defined the \textbf{rank} of a matrix as the number of pivot columns in RREF. How does the rank of \textbf{G} relate to the column space of \textbf{G}? (\textit{Hint: What property of a space gives a single value out?} )

\vspace{1in}

\Large Note this relationship between rank and column spaces is actually true for any matrix. \normalsize

\vspace{0.2in}

7) Give a basis for the column space of $\textbf{F}$ (from problem 2).

\newpage 

\section*{Rank-Nullity Theorem}
8) You should have already found a basis for each of the column space and the null space of $\textbf{F}$ (problems 2 and 7).

\vspace{0.2in}

9a) What are the dimensions of the null space and column space for \textbf{F}?

\vspace{0.75in}

9b) How do the dimensions of \textbf{F} relate to the sum: dim(column space of \textbf{F}) + dim(null space of \textbf{F})?

\vspace{1in}

10a) State the dimension of the null space (the \textit{nullity} of \textbf{G}) and the dimension of the column space for matrix \textbf{G}.

\vspace{0.75in}

10b) How do the dimensions of \textbf{G} relate to the sum: dim(column space of \textbf{G}) + dim(null space of \textbf{G})?

\vspace{1in}

11) The \textbf{Rank-Nullity Theorem} generalizes the above results for the $m\times n$ matrix \textbf{A}. Based on your results, what do you think the \textbf{Rank-Nullity Theorem} is?

\vspace{0.75in}

\end{flushleft}
\end{document}