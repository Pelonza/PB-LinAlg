\documentclass{article}
\pagestyle{empty}
\usepackage{amsmath,amssymb,amsfonts}
\usepackage{graphicx}
\usepackage{multicol}
\setlength{\oddsidemargin}{0in} \setlength{\evensidemargin}{0in}
\setlength{\topmargin}{0in} \setlength{\textheight}{8.5in}
\setlength{\textwidth}{6.5in}

\makeatletter
\renewcommand*\env@matrix[1][*\c@MaxMatrixCols c]{%
	\hskip -\arraycolsep
	\let\@ifnextchar\new@ifnextchar
	\array{#1}}
\makeatother

\begin{document}
\begin{flushleft}
	\bfseries{MATH 260, Linear Algebra, Spring `14}\\
	\bfseries{Activity 9:  Eigenvalues and Eigenvectors}\\
	\bfseries{Honor Code:} \hspace{3.5in}\bfseries{Names:}\\
\end{flushleft}
\begin{flushleft}
\vspace{.75in}
Directions:  Everyone should work on the assignment and should fill out their paper.  You are expected to make corrections based on what is presented on the board.  \\
\large The worksheet is due by the end of the day (5pm). \normalsize \\ 
If you need more explanations after class, you can read Section 5.3 of your textbook.\\
\vspace{0.1in}
\Large
In-Class Learning Goals:\\
\normalsize
\begin{enumerate}
\item Be able to calculate the eigenvalues and eigenvectors of a 2x2 matrix.
\item Understand how to calculate the eigenvalues and eigenvectors for ANY matrix.
\item (Stretch) Be able to identify several properites of eigenvalues/vectors.
%\item Be able to determine if a set of vectors is a basis for a vector space
%\item (stretch) Be able to find the dimension of a set or space (by the end of homework for sure!)

\end{enumerate}

\vspace{0.1in}

\section*{Warm-up:  Matrix Multiplication}

Given the matrix 
$\textbf{A}=
\begin{bmatrix} 
1 & 1 \\ 4&1 
\end{bmatrix}$
 and the vectors: 
$\vec{v}_1=\begin{bmatrix} 2 \\ 2 \end{bmatrix}$, 
$\vec{v}_2=\begin{bmatrix} 1 \\ 2 \end{bmatrix}$,
$\vec{v}_3=\begin{bmatrix} -2 \\ 0 \end{bmatrix}$, 
$\vec{v}_4=\begin{bmatrix} -1 \\ -4 \end{bmatrix}$, 
$\vec{v}_5=\begin{bmatrix} 1 \\ -2 \end{bmatrix}$.

\vspace{0.1in}

Find $\textbf{A}\vec{v}_i$ for $i=1\rightarrow 5$. 

\vspace{2in}

\newpage

\section*{Visualizing Eigenvectors}

1. Sketch each $\vec{v}_i$ and the result of $\textbf{A}\vec{v}_i$ from the warm-up on the x-y plane. You should have 5 pictures when you are done. Be sure to clearly label which was your initial, and which your resultant vector. 

\vspace{4in}

2. Two vectors behaved differently, which ones? Describe \& discuss at your table how they behaved differently.

\vspace{1in}

3. For the two vectors that behaved differently, you should be able to write the eigenvector equation: $\textbf{A}\vec{v}=\lambda \vec{v}$. Find $\lambda_i$ for each pair of vectors.

\vspace{.75in}

\newpage

\section*{Practice}

Let's practice.

\vspace{0.1in}

4. For each matrix, take the determinant $|\textbf{A}- \lambda \textbf{I}  | $ then set the resulting polynomial equal to 0 and solve for $\lambda$. (they should all be factorable). Once you have found ALL the eigenvalues ($\lambda$) find at least one eigenvector for each matrix.\\

\begin{center}

$\textbf{W}=\begin{bmatrix}[rr] 1 & 4\\-4&11 \end{bmatrix}$
\hspace{0.3in}
$\textbf{G}=\begin{bmatrix}[rrr] 1 & 1 & -2\\ -1 & 2 & 1\\ 0 & 1 & -1 \end{bmatrix}$
\hspace{0.3in}
$\textbf{R}=\begin{bmatrix}[rrr] 1 & 2 & 3 \\ 0 & 4 & 5 \\ 0 & 0 & 6 \end{bmatrix}$

\end{center}

\vspace{6in}

5. Go back and look at $\textbf{R}$ what do you notice about the eigenvalues you found for \textbf{R}?

\newpage

\section*{Some Properties of Eigenvalues}

\textit{ Note: We will find these for $2 \times 2$ matrices, but these properties generalize to any $n \times n$ matrix. }

\vspace{0.3in}

6. We find eigenvalues by solving: $|\textbf{A}- \lambda \textbf{I}  | =0 $. What happens if $\lambda = 0 $? What must be true about the matrix \textbf{A} for \textbf{A} to have eigenvalues of 0?

\vspace{1in}

7. (a) Find $\textbf{W}^{T}$.

\vspace{1in}

\hspace{0.12in} (b) What is the polynomial from $|\textbf{W}^{T}- \lambda \textbf{I}  | $ and how does it compare to the polynomial for $|\textbf{W}- \lambda \textbf{I}  | $?

\vspace{1in}

\hspace{0.12in} (c) What can you say about the eigenvalues (and vectors) of $\textbf{A}$ and $\textbf{A}^{T}$?

\vspace{0.75in}

8. (a) Find $\textbf{W}^{-1}$. (the inverse)

\vspace{1in}

\hspace{0.12in} (b) What is the polynomial from $|\textbf{W}^{-1}- \lambda \textbf{I}  | $? Set the polynomial equal to 0, and solve for $\lambda$

\vspace{1in}

\hspace{0.12in} (c) Look carefully, how do the eigenvalues of $\textbf{W}^{-1}$ relate to the eigenvalues of \textbf{W}?


%
%As you might have guessed from the notation above (or remembered from your textbook) you have identified the eigenvectors and eigenvalues for the matrix \textbf{A}. Multiplication by a matrix is called a \textit{linear transformation}. We can more generally write the results from above using the equation:\\
%\begin{equation*} \textbf{A}\vec{v}=\lambda \vec{v} \end{equation*}
%
%\vspace{0.1in}
%
%\section*{Computing Eigenvectors}
%
%Let's take a closer look at the equation: $\textbf{A}\vec{v}=\lambda \vec{v}$ We've learned before we can do algebric manipulation on matrices (this is linear algebra after all), so here's a series of manipluations:
%\begin{center}
%$\textbf{A}\vec{v}=\lambda \vec{v} \rightarrow \textbf{A}\vec{v} - \lambda \vec{v} = 0 \rightarrow (\textbf{A} -\lambda ) \vec{v} = 0$
%\end{center}
%Remember, above we found that $\lambda$ was just a scalar value...
%
%\vspace{0.1in}
%
%4. If \textbf{A} was a $2 \times 2$ matric, can you find $\textbf{A}-\lambda$? What about if it was a $3 \times 3$ matrix or an $n \times n$ matrix. If so, what is it, if not, why not? \\
%\textit{If it helps to have a number for thinking, you can pick either $\lambda$ you found above}
%
%\vspace{1in}
%
%5. Can you find $\textbf{A}- \lambda \textbf{I}$ (for \textbf{A} as a $2 \times 2$, $3 \times 3$, or $n \times n$ matrix)? If so, what is it, if not, why not? \\
%\textit{If it helps to have a number for thinking, you can pick either $\lambda$ you found above}
%
%\vspace{1in}
%
%6. If you haven't talked about your answers to 4. and 5. at your table, or with the professor, do so now.
%
%\vspace{0.2in}
%
%7. Now, it is time to use your system solving skills (finding RREF) to find eigenvectors. \\
%Create an augmented matrix (for each $\lambda$) then solve the system: $(\textbf{A}- \lambda \textbf{I}) \begin{bmatrix} v_1\\v_2\end{bmatrix}=\vec{0}$
%
%\vspace{2.5in}
%
%8. How many solutions did each system have? How does that correspond to the special trait you observed in 2?
%
%\vspace{1in}
%
%9. You already knew the eigenvectors for \textbf{A} (they are the two special vectors from before), how do these relate to the RREF form of the augmented matrices?
%
%\vspace{1in}
%
%\section*{Computing Eigenvalues}
%
%Using $(\textbf{A}- \lambda \textbf{I}) \vec{v}=\vec{0}$ to find the eigenvectors is great, but it only works if we already know the eigenvalues. We need to use the same equation (in a different way) to find the eigenvalues. Let's examine the equation a little closer...
%
%\vspace{0.2in}
%
%10. Let $x= (\textbf{A}- \lambda \textbf{I})$ and $ y = \vec{v}$. Then we have the equation: $x y = 0$. If we let $x$ take any value, what must $y$ equal for the equation to be true? What about $x$ if we let $y$ take any value?\\(Show why or write a sentence, don't just write the value.)
%
%\vspace{1in}
%
%11. Would it be useful to have $y=\vec{v}=0$ (that is, eigenvectors equal to 0)? Why or why not?\\
%It might help to think about how the $\vec{\textbf{0}}$ relates to a vector space and basis. \\
%(You don't need an essay, but discuss at your table and write down at least a sentence or two)
%
%\vspace{1in}
%
%If we want  $x= (\textbf{A}- \lambda \textbf{I})=0$ this is only true if $|\textbf{A}- \lambda \textbf{I}  | = 0$. Solving the polynomial equations generated from this determinant is our general method for finding eigenvalues. 
%
%\vspace{0.2in}
%
%\newpage


\end{flushleft}
\end{document}