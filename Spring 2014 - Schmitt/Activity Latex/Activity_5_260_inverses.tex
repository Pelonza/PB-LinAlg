\documentclass{article}
\pagestyle{empty}
\usepackage{amsmath,amssymb,amsfonts}
\usepackage{graphicx}
\usepackage{multicol}
\setlength{\oddsidemargin}{0in} \setlength{\evensidemargin}{0in}
\setlength{\topmargin}{0in} \setlength{\textheight}{8.5in}
\setlength{\textwidth}{6.5in}

\makeatletter
\renewcommand*\env@matrix[1][*\c@MaxMatrixCols c]{%
	\hskip -\arraycolsep
	\let\@ifnextchar\new@ifnextchar
	\array{#1}}
\makeatother

\begin{document}
\begin{flushleft}
	\bfseries{MATH 260, Linear Algebra, Spring `14}\\
	\bfseries{Activity 5:  Matrix Inverses}\\
	\bfseries{Honor Code:} \hspace{3.5in}\bfseries{Names:}\\
\end{flushleft}
\begin{flushleft}
\vspace{.75in}
Directions:  Everyone should work on the assignment and should fill out their paper.  You are expected to make corrections based on what is presented on the board.  \\
\begin{center} \large The worksheet is due by the end of the day (5pm). \normalsize \\

\vspace{0.1in}
\Large
In-Class Learning Goals:\\
\normalsize
\begin{enumerate}
\item Be able to find the inverse of a matrix.
\item Be able to identify several characteristics of an invertible matrix. 
\item Be able to state the number of solutions to a matrix-vector equation, based on invertibility.
\end{enumerate}
\vspace{0.1in}
\section*{Warmup: RREF Review}
\end{center}

Find the RREF of the following matrix:\\
\begin{center}
$\begin{bmatrix}[cc|cc]
1 & 1 &  1 & 0 \\
4 & 1 &  0 & 1 \\

\end{bmatrix}
$\\
\end{center}
\vspace{2in}
Congratulations! You've found your first matrix inverse.
\newpage
\begin{center}
\section{The Identity Matrix and Inverses}
\end{center}
Notice that the right-hand side (RHS) of the warmup started as the identity matrix. The warmup exercise could be stated as: $[\textbf{A}|\textbf{I}]$. Reducing this combined matrix into RREF is the simplest way to find the inverse of a matrix. The inverse of the matrix \textbf{A} is the RHS after getting the left-side into RREF.\\
\vspace{0.1in}
a) One trait of the inverse is that the statement: $\textbf{A} \textbf{A}^{-1} = \textbf{I} $. Verify this is true for the warmup.\\
\vspace{1.5in}
b) Another trait if a matrix has an inverse, then its transpose has an inverse. Let's check that:\\
(i) Find $\textbf{A}^T$. Recall that the transpose is defined as: $[a_{ij}]^T = [a_{ji}]$\\
\vspace{1.5in}
(ii) Find the inverse of $\textbf{A}^T$. \textit{Hint: start with the matrix  $\left[ A^T | I \right] $ }\\
\vspace{2in}
c) Another practice. Obtain the inverse of the following matrix using row operations and RREF.\\
\begin{center}
$\begin{bmatrix}[ccc]
1 & 1 & 1 \\
0 & 2 & 1 \\
1 & 0 & 1 
\end{bmatrix}$

\newpage
\section{A Use for the Inverse}
\end{center}
a) How did we use RREF to find solutions to the matrix-vector equation: $\textbf{A} \vec{\textbf{x}} = \vec{\textbf{b}}$?\\
\vspace{1in}
b) If we do the same operations to find a RREF, can't we use the inverse to find an answer too...?\\
(i) Solve the following system by finding the RREF of the \textbf{\textit{augmented}} matrix:\\
\begin{center}
$\begin{array}{rrrrr}
x & + & y & = & -1\\
4x & + & y & = & -7
\end{array}
$ \\ \end{center}
\vspace{2in}
(ii) Now multiply the inverse you found in the warmup by $\vec{\textbf{b}}=\begin{bmatrix} -1 \\ -7 \end{bmatrix}$ \\
\vspace{1.5in}
c) Discuss your results at your table, and write down observations. Here's a little guidance... \\
Left multiply each side of the equation $\textbf{A}\vec{\textbf{x}}=\vec{\textbf{b}}$ by $\textbf{A}^{-1}$ and compare to answers you've obtained so far...
\vspace{2in}
\newpage
\begin{center}
\section{More Solutions...}
\end{center}
a) Find the inverse of the matrix:
\begin{center}
$\textbf{C} = 
\begin{bmatrix}[rrr]
3 & 0 & 3\\
-1 & 2 & 1\\
1 & 1 & 2
\end{bmatrix}
$\\\end{center}
\vspace{2in}
b) What happened? Based on your exploration in Section 2, if this matrix described a system, do you think it has a unique answer? Support your response, and discuss at your table. (This is a good place to check in once you've discussed)\\
\vspace{2in}
\Large
Invertibility and Solutions (pg 151)\\
\normalsize
Notes:\\
\vspace{1in}
\begin{center}
\section{Challenge}
\end{center}
Find a constant $a$ so each of the following matrices are invertible. If no such $a$ exists, says so and explain why. (Problems 32 \& 33 from pg 155)\\
\begin{center}
$\textbf{H}= \begin{bmatrix}
1 & 0 & a\\
0 & 1 & 0\\
0 & 0 & 1
\end{bmatrix} $
\hspace{1in}
$\textbf{T}= \begin{bmatrix}
1 & 0 & 1\\
0 & 1 & 0\\
a & a & a
\end{bmatrix} $
\end{center}

\end{flushleft}
\end{document}