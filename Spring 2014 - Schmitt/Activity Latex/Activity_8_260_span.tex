\documentclass{article}
\pagestyle{empty}
\usepackage{amsmath,amssymb,amsfonts}
\usepackage{graphicx}
\usepackage{multicol}
\setlength{\oddsidemargin}{0in} \setlength{\evensidemargin}{0in}
\setlength{\topmargin}{0in} \setlength{\textheight}{8.5in}
\setlength{\textwidth}{6.5in}

\makeatletter
\renewcommand*\env@matrix[1][*\c@MaxMatrixCols c]{%
	\hskip -\arraycolsep
	\let\@ifnextchar\new@ifnextchar
	\array{#1}}
\makeatother

\begin{document}
\begin{flushleft}
	\bfseries{MATH 260, Linear Algebra, Spring `14}\\
	\bfseries{Activity 8:  Linear Independence, Basis, Spans (and Dimension)}\\
	\bfseries{Honor Code:} \hspace{3.5in}\bfseries{Names:}\\
\end{flushleft}
\begin{flushleft}
\vspace{.75in}
Directions:  Everyone should work on the assignment and should fill out their paper.  You are expected to make corrections based on what is presented on the board.  \\
\large The worksheet is due by the end of the day (5pm). \normalsize \\ 
If you need more explanations after class, you can read Section 3.6 of your textbook.\\
\vspace{0.1in}
\Large
In-Class Learning Goals:\\
\normalsize
\begin{enumerate}
\item Be able to determine if a set of vectors is linearly independent
\item Be able to explain the span of a set, the relationship between span and the basis of a vector space
\item Be able to determine if a set of vectors is a basis for a vector space
\item (stretch) Be able to find the dimension of a set or space (by the end of homework for sure!)

\end{enumerate}

\vspace{0.1in}

\section*{Warm-up:  Linear Combinations}

Let's look at the vectors: $\vec{c}=<1,0,0>$, $\vec{d}=<-1,1,1>$ and $\vec{e}=<0,2,2>$. 

\vspace{0.1in}

1. Write $\vec{e}$ as a linear combination of $\vec{c}$ and $\vec{d}$.

\vspace{1.5in}

2. Can you do the same thing for each other vector? That is, write each vector as a linear combination of the other two vectors. Give the combinations, or show you cannot.

\newpage
\section*{Linear Independence}
 \textbf{Definition:} A set of vectors $ \{ \vec{v}_1 , \vec{v}_2, \cdots \vec{v}_n \}$ is \textit{linearly independent} if no vector of the set can be written as a linear combination of the other vectors. Otherwise it is \textit{linearly dependent}

\vspace{0.1in}

1. a) Is the set of vectors $\{ <1,0,0>, <-1,1,1>, <0,2,2> \}$ linearly independent? Explain.

\vspace{1.5in}

%Another way to think about independence is to say that the set of vectors $ \{ \vec{v}_1 , \vec{v}_2, \cdots \vec{v}_n \}$ is \textit{linearly independent} if and only if when:\\
%\begin{eqnarray*}
%c_1 \vec{v}_1 + c_2 \vec{v}_2 + ... + c_n \vec{v}_n = \vec{0}\\
%\text{ then : } c_1=c_2=...=c_n= 0
%\end{eqnarray*}
%Written in matrix-vector form, it looks like: $[ \vec{v}_1,\vec{v}_2,\cdots \vec{v}_n ] \left[ c_1\\c_2\\\vdots\\c_n \right]=\vec{0}$. Let $\textbf{A} = [ \vec{v}_1,\vec{v}_2,\cdots \vec{v}_n ] $ and $\vec{x}= \left[ c_1\\c_2\\\vdots\\c_n \right] $ \\
%\vspace{0.1in}
%A property of invertible matrices we haven't talked about is that if a matrix is invertible, then the equation $\textbf{A}\vec{x}=\vec{0}$ only has the solution $\vec{x}=\vec{0}$.
%\vspace{0.1in}

1. b) Write the vectors $\{ <1,0,0>, <-1,1,1>, <0,2,2> \}$ as a matrix with each vector as a vertical column. Find the determinant of the matrix.

\vspace{1.5in}

2. a)Take the vectors $\vec{a}=<1,2>$ and $\vec{b}=<1,3>$. Write these as a matrix, then take the determinant.

\vspace{1.5in}

2. b) For what values of $c_1,c_2$ is the following equation true?\\ \begin{equation*} c_1 \vec{a} + c_2 \vec{b} = \vec{0} \end{equation*}

\vspace{1in}

3. a)Put each matrix into RREF, then give the number of pivot columns:\\
$\begin{bmatrix}
1 & -1 & 0\\
0 & 1&2\\
0&1&2
\end{bmatrix}$
\hspace{0.5in}
$\begin{bmatrix}
1 & 1\\
2 & 3
\end{bmatrix}$
\hspace{0.5in}
$\begin{bmatrix}
1 & 2\\
2 & 4
\end{bmatrix}$

\vspace{3in}

3. b) Are the vectors used in the third matrix linearly independent? Explain why or why not.

\vspace{1in}

3. c) How do the number of pivot columns compare to the number of elements in the vectors? How does this correspond to if the vectors are linearly independent? How does the value of the determinant correspond to if vectors are linearly independent?

\vspace{1.5in}

4. a) You now have 3 ways of testing for linear independence, what are they?

\vspace{1in}

4. b) You actually know a 4th way, recall that if matrix is invertible, then $|A|\neq 0$. \\
i) How many (non-zero) column vectors of length 2 do you need to make a $2 \times 3$ matrix?\\
ii) Can you find the inverse of a $2\times 3 $ matrix? Can you find the determinant?\\
iii) If a set of vectors has more vectors in it then the length of the vectors, will the set be linearly independent? (i.e. if the set contains 3 vectors of length 2, or 4 vectors of length 2 or 3)

\vspace{1in}

\section*{Basis}
A set is a \textbf{basis} of a vector space if it has the properties:\\
(i) The set is linearly independent\\
(ii) The span of the set covers the entire vector space.\\
\vspace{0.1in}
1. Let's test some of the vector sets we've looked at. Which set of vectors from the previous section might constitute a basis? Explain why.

\vspace{1in}

2. To satisfy the 2nd property, we need the span of these vectors to cover the entire vector space of $\mathbb{R}^2 $ This means we need to be able to create each of the following vectors via linear combinations:\\
\begin{itemize}
\item A vector with all positives: $<+,+>$
\item A vector with all negatives: $<-,->$
\item A vector with a negative and a positive in each position: $<+,->$ and $<-.+>$
\item The zero vector: $<0,0>$
\end{itemize}
Write a linear combination of $\vec{a}$ and $\vec{b}$ which gives each of the above types of vectors.

\vspace{2in}

3. a) We know the vectors from above $\vec{c}$, $\vec{d}$, $\vec{e}$ are not linearly independent. What if we just use the vectors $\vec{c}=<1,0,0>$ and $\vec{d}=<-1,1,1>$ do we have a basis for $\mathbb{R}^3$?\\
i) Check if $\vec{c}$ and $\vec{d}$ are linearly independent\\
ii) Check if the span\{ $\vec{c}$, $\vec{d}$ \} contains all the vectors possible in $\mathbb{R}^3$. If it does not, give an example of a vector that it does not contain.

\vspace{3in}

3. b) Your span did not contain all of $\mathbb{R}^3$. What vector could you add to the set that makes the new set be a basis for $\mathbb{R}^3$? (show that the vector you choose is both linearly independent AND allows the span to contain all of $\mathbb{R}^3$)

\vspace{2in}


\end{flushleft}
\end{document}