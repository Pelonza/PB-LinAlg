\documentclass{article}
\pagestyle{empty}
\usepackage{amsmath,amssymb,amsfonts}
\usepackage{graphicx}
\usepackage{multicol}
\setlength{\oddsidemargin}{0in} \setlength{\evensidemargin}{0in}
\setlength{\topmargin}{0in} \setlength{\textheight}{8.5in}
\setlength{\textwidth}{6.5in}

\makeatletter
\renewcommand*\env@matrix[1][*\c@MaxMatrixCols c]{%
	\hskip -\arraycolsep
	\let\@ifnextchar\new@ifnextchar
	\array{#1}}
\makeatother

\begin{document}
\begin{flushleft}
	\bfseries{MATH 260, Linear Algebra, Spring `14}\\
	\bfseries{Activity 11:  Diagonalization}\\
	\bfseries{Honor Code:} \hspace{3.5in}\bfseries{Names:}\\
\end{flushleft}
\begin{flushleft}
\vspace{.75in}
Directions:  Everyone should work on the assignment and should fill out their paper.  You are expected to make corrections based on what is presented on the board.  \\
If you need more explanations after class, you can read Section 5.4 of your textbook.\\
\vspace{0.1in}
\Large
In-Class Learning Goals:\\
\normalsize
\begin{enumerate}
\item Be able diagonalize a matrix
\item Be able to identify matrices that cannot be diagonalized
\item (Stretch/Homework) Be able to use the diagonalization of a matrix to find a matrix power or check similarity
%\item Be able to determine if a set of vectors is a basis for a vector space
%\item (stretch) Be able to find the dimension of a set or space (by the end of homework for sure!)

\end{enumerate}

\vspace{0.1in}

\section*{Warm-up:  Eigenvectors and Spaces of Repeated Roots}
%This is the matrix from Sect. 5.4, pg 339, #43

$\textbf{G}
=\begin{bmatrix}
1 & 0 & 0\\
-4 & 3 & 0 \\
-4 & 2 & 1
\end{bmatrix}
$
\hspace{0.5in}
The eigenvalues of \textbf{G} are $\lambda_{1,2}=1$ and $\lambda_3 = 3$. 

\vspace{0.25in}
1. If I had asked you to find the eigenvalues of \textbf{G}, you would NOT need to calculate $det(\textbf{G}-\lambda \textbf{I})=0$. Explain why.

\vspace{1in}

2. Find the three eigenvectors associated with the given eigenvalues.

\vspace{3in}

3. What is the dimension of each corresponding eigenspace? What is the sum of the eigenspace dimensions for \textbf{G}?

\vspace{1in}

\section*{Diagonalized Equations}
Define two matrices: $\textbf{P} = \begin{bmatrix}
| & | & & |\\
\vec{v}_1 & \vec{v}_2 & \ldots &\vec{v}_n \\
| & | & & |
\end{bmatrix} $
\hspace{0.5in}
$\textbf{D} = \begin{bmatrix}
\lambda_1 & 0 & \ldots & 0 \\
0 & \lambda_2 & \ldots & 0 \\
\vdots & \vdots & \ddots & \vdots \\
0 & 0 & \ldots & \lambda_n 
\end{bmatrix} $

\vspace{0.1in}

We are going to check some matrix equations using \textbf{P}, $\textbf{P}^{-1}$ and \textbf{D}. 
For each of the following be sure to show some intermediate steps, since we already know what the final result should be!

\vspace{0.1in}

For \# 4-8 use the matrix: $\textbf{A} = \begin{bmatrix} 1 & 1 \\ 4 & 1 \end{bmatrix}$ \hspace{0.2in} with eigenvalues $\lambda_1 = 3$ and $\lambda_2 = -1$ with corresponding eigenvectors $\vec{v}_1 = \begin{bmatrix} 1 \\ 2 \end{bmatrix}$ and $\vec{v}_2 = \begin{bmatrix} 1 \\ -2 \end{bmatrix}$ respectively.

\vspace{0.2in}

4. Find $\textbf{AP}$ and $\textbf{PD}$. How do they compare?

\vspace{2.5in}

5. Find $\textbf{PDP}^{-1}$. What does it equal that we have already defined? \\
\textit{Hint: You already calculated $\textbf{PD}$ you just need to find $\textbf{P}^{-1}$, and do one multiplication}

\vspace{2.5in}

6. Find $\textbf{P}^{-1} \textbf{AP}$. What does it equal that we have already defined?\\
\textit{Hint: You only need to do ONE new multiplication here!}

\vspace{2in}

7. Write all three of the equations you have calculated above here, using the variables \textbf{A}, \textbf{P}, \textbf{D}, and $\textbf{P}^{-1}$. \\
\textit{This is a good place to check with your table, and professor to be sure everything came out right!}

\vspace{1in}

8. a) Define the matrix $\textbf{F}=\begin{bmatrix} \lambda_2 & 0 \\ 0 & \lambda_1 \end{bmatrix} = \begin{bmatrix} -1 & 0 \\ 0 & 3 \end{bmatrix}$, and use the same \textbf{P} as above. Find $\textbf{PFP}^{-1}$. 

\vspace{2.5in}

8. b) Does it matter to the equations you wrote in \# 6 what order the eigenvalues and eigenvectors are in \textbf{P} and \textbf{D}? What relationship is required between \textbf{P} and \textbf{D}?

\vspace{1in}

\section*{Diagonalizing with Repeated Eigenvalues}
9. In the warm-up you found the eigenvectors for a matrix with a repeated eigenvalue, but which had 3 linearly independent eigenvectors. Pick your favorite equation from \# 6. Is the equation valid for \textbf{G}?

\vspace{2.5in}

%The matrix below is from Sect 5.4, pg 339, #29
10. The matrix $\begin{bmatrix} 3 & 1 \\ -1 & 5 \end{bmatrix}$ has a repeated eigenvalue of 4, with one eigenvector of $\vec{v}_1=\begin{bmatrix} 1 \\ 1 \end{bmatrix}$. \\
If we let $\textbf{P}=\begin{bmatrix} 1 & 1 \\ 1 & 1 \end{bmatrix}$ and $\textbf{D} = \begin{bmatrix} 4 & 0 \\ 0 & 4 \end{bmatrix}$. Does the equation $\textbf{A} = \textbf{PDP}^{-1}$ still hold? \\

\textit{ Hint: Think about \textbf{P}. What is $det(\textbf{P})$? What does that tell you about $\textbf{P}^{-1}$? }

\vspace{2in}

11. If the results from \# 9 and \# 10 are generalizable (they are), what must be true about a matrix for us to diagonalize it?

\vspace{1in}

%\section*{Applications of Diagonalization: Matrix Powers}
%We are again going to use the matrix: $\textbf{A} = \begin{bmatrix} 1 & 1 \\ 4 & 1 \end{bmatrix}$ \hspace{0.2in} with eigenvalues $\lambda_1 = 3$ and $\lambda_2 = -1$ with corresponding eigenvectors $\vec{v}_1 = \begin{bmatrix} 1 \\ 2 \end{bmatrix}$ and $\vec{v}_2 = \begin{bmatrix} 1 \\ -2 \end{bmatrix}$ respectively.
%
%\vspace{0.1in}
%
%12. Compute $\textbf{PD}^2 \textbf{P}^{-1}$ and $\textbf{A}^2 $. What do you notice?
%
%\vspace{2.5in}
%
%13. Compute $\textbf{PD}^3 \textbf{P}^{-1}$. Is it equal to $\textbf{A}^3 = \begin{bmatrix} 13 & 7 \\ 28 & 13 \end{bmatrix}$?
%
%\vspace{2in}
%
%14. Give a general equation to compute $\textbf{A}^k$.
%
%\vspace{1in}

%\subsection*{Similar Matrices}
%For this problem we'll use the matrices:\\
%$\textbf{X}=\begin{bmatrix} 4 & -2 \\ 1 & 1 \end{bmatrix}$ 
%\hspace{0.25in}
%$\textbf{Y}=\begin{bmatrix} -3 & 10 \\ -3 & 8 \end{bmatrix}$\\
%Their associated \textbf{P} matrices are: \\
%$\textbf{P}_X = \begin{bmatrix} 2 & 1 \\ 1 & 1 \end{bmatrix}$
%\hspace{0.25in}
%$\textbf{P}_Y = \begin{bmatrix} 5 & 2 \\ 3 & 1 \end{bmatrix}$
%
%\vspace{0.2in}
%
%15. Find $\textbf{P}_X \textbf{X} \textbf{P}^{-1}_X$ and $\textbf{P}_Y \textbf{Y} \textbf{P}^{-1}_Y$.
%
%\vspace{3in}
%
%16. Find the trace and determinant of \textbf{X} and \textbf{Y}.


\end{flushleft}
\end{document}