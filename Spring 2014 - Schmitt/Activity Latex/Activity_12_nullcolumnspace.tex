\documentclass{article}
\pagestyle{empty}
\usepackage{amsmath,amssymb,amsfonts}
\usepackage{graphicx}
\usepackage{multicol}
\setlength{\oddsidemargin}{0in} \setlength{\evensidemargin}{0in}
\setlength{\topmargin}{0in} \setlength{\textheight}{8.5in}
\setlength{\textwidth}{6.5in}

\makeatletter
\renewcommand*\env@matrix[1][*\c@MaxMatrixCols c]{%
	\hskip -\arraycolsep
	\let\@ifnextchar\new@ifnextchar
	\array{#1}}
\makeatother

\begin{document}
\begin{flushleft}
	\bfseries{MATH 260, Linear Algebra, Spring `14}\\
	\bfseries{Activity 12:  Null \& Column Spaces}\\
	\bfseries{Honor Code:} \hspace{3.5in}\bfseries{Names:}\\
\end{flushleft}
\begin{flushleft}
\vspace{.75in}
Directions:  Everyone should work on the assignment and should fill out their paper.  You are expected to make corrections based on what is presented on the board.  \\
If you need more explanations after class, you can read Section 5.4 of your textbook.\\
\vspace{0.1in}
\Large
In-Class Learning Goals:\\
\normalsize
\begin{enumerate}
\item Be able to find the null space/kernel of any matrix
\item Be able to give the column space for any matrix
\item Be able to give a basis for the column space of a matrix
\item Be able to describe how the null space and column space relate to each other
%\item Be able to determine if a set of vectors is a basis for a vector space
%\item (stretch) Be able to find the dimension of a set or space (by the end of homework for sure!)

\end{enumerate}

\vspace{0.1in}

\section*{Warm-up:  In the null space?}
%This is the matrix from Sect. 5.4, pg 339, #43
Recall that the null space is all vectors $\vec{x}$ such that $\textbf{A}\vec{x}=\vec{0}$. Given the matrix:\\
\begin{center}
$\textbf{G}=
\begin{bmatrix}
2 & 3 & -1\\
-1 & 4 & 6 \\
1 &  7 & 5
\end{bmatrix}
$\\
\end{center}
1a) Determine which of the following vectors are in the null space of \textbf{G}.\\
\begin{center}
$ \vec{w}_1=\begin{bmatrix} 3 \\ 2 \\ 1 \end{bmatrix} $
\hspace{0.3in}
$ \vec{w}_2=\begin{bmatrix} 8 \\ -4 \\ 4 \end{bmatrix} $
\hspace{0.3in}
$ \vec{w}_3=\begin{bmatrix} -2 \\ 1 \\ -1 \end{bmatrix} $
\end{center}

\vspace{2in}

1b) Find the actual null space of \textbf{G}. Do your answers above make agree?

\newpage

\section*{A useful Null Space}
Given the matrix $\textbf{F} = \begin{bmatrix} 1 & 2 \\ 3 & 0 \end{bmatrix}$ with eigenvalues of -2 and 3.

\vspace{0.1in}

2) Find the null space of the matrix $(\textbf{F} + 2 \textbf{I} )$. 

\vspace{1.25in}

3) Find the null space of the matrix $(\textbf{F} - 3 \textbf{I} )$. 

\vspace{1.25in}

4) Have you seen these null spaces (or their calculations) before? What do the null spaces you found in (2) and (3) have to do with the matrix \textbf{F}?

\vspace{1in}

\section*{Basis for Column Space}
We know that the column space ( $span( \{ \vec{v}_1 , \vec{v}_2, \ldots \vec{v}_n \} )$ ) is a subspace. However, generally in mathematics we want to describe things in as simple terms as possible. For spaces, this means giving a \textit{basis} only for a space. \\
Recall that a set is a \textbf{basis} of a vector space if it has the properties:\\
(i) The set is linearly independent \hspace{.25in} (ii) The span of the set covers the entire vector space.\\

\vspace{0.1in}

5) The column space of \textbf{G} could be given as \textit{span}$ \left( \left\{ \begin{bmatrix} 2 \\ -1 \\ 1\end{bmatrix}, \begin{bmatrix} 3 \\ 4 \\ 7 \end{bmatrix}, \begin{bmatrix} -1 \\ 6 \\ 5 \end{bmatrix} \right\} \right)$. Does the set $S=\left\{ \begin{bmatrix} 2 \\ -1 \\ 1 \end{bmatrix}, \begin{bmatrix} 3 \\ 4 \\ 7 \end{bmatrix}, \begin{bmatrix} -1 \\ 6 \\ 5 \end{bmatrix} \right\}$ form a basis for the column space? (Show why or why not)

\vspace{1.5in}

We already have all the tools and information to define a valid basis. Let's do so...

\vspace{0.1in}

6) Which columns in the RREF of \textbf{G} are pivot columns?

\vspace{0.75in}

7) Take the columns from the original \textbf{G} that you identified as the pivot columns. Form a new set $S_{B}$ from these columns. These should be $3 \times 1$ vectors with non-zero values in each row.  \textit{Be sure to show work or explain your reasoning below, a yes/no answer for any of the below will NOT receive credit.}

\vspace{0.1in}

7a) Is this new set $S_{B}$ linearly independent?

\vspace{1.5in}

7b) Does it span the entire column space? ( \textit{Hint: Does $span( S ) = span ( S_{B} ) $ } )

\vspace{1.5in}

7c) Is the set $S_{B}$ a basis for the column space? Explain.

\vspace{0.75in}

8) We defined the \textbf{rank} of a matrix as the number of pivot columns in RREF. How does the rank of \textbf{G} relate to the column space of \textbf{G}? (\textit{Hint: What property of a space gives a single value out?} )

\vspace{1.6in}

\Large Note this relationship between rank and column spaces is actually true for any matrix. \normalsize

\newpage 

\section*{Rank-Nullity Theorem}
9) For the CPA you found the null space and column-space of $\textbf{A}=\begin{bmatrix} 1 & -2 \\ 1 & -2 \end{bmatrix}$.\\
A basis for the column space is $\begin{bmatrix} 1 \\ 1 \end{bmatrix}$. A basis for the null space is $\begin{bmatrix} 2 \\ 1 \end{bmatrix}$.

\vspace{0.15in}

9a) What is the dimension of the null space and column space for \textbf{A}?

\vspace{0.75in}

9b) How do the dimensions of \textbf{F} relate to the sum: dim( column space of \textbf{A}) + dim( null space of \textbf{A} ) ?

\vspace{1in}

10a) State the dimension of the null space (the \textit{nullity} of \textbf{G} ) and the dimension of the column space for matrix \textbf{G}.

\vspace{0.75in}

10b) How do the dimensions of \textbf{G} relate to the sum: dim( column space of \textbf{G} ) + dim( null space of \textbf{G}) ?

\vspace{1in}

11) The \textbf{Rank-Nullity Theorem} generalizes the above results for the $m\times n$ matrix \textbf{A}. Based on your results, what do you think the \textbf{Rank-Nullity Theorem} is? (Take a guess BEFORE looking this up! You should check with your professor or check your text once you make a guess.)

\vspace{0.75in}

%\section*{Row-Spaces and Left-Nullspace}
%Just as each matrix has a column-space, it also has a row-space. These row spaces are defined analogously. While we could define a whole new set of column operations there's an easier way...
%
%\vspace{0.1in}
%
%7a) Find the transpose of matrix \textbf{G} $\rightarrow \textbf{G}^{T}$.
%
%\vspace{1in}
%
%7b) How are the rows of \textbf{G} related to the columns of $\textbf{G}^T$?
%
%\vspace{1in}
%
%7c) Find a basis for the column-space of $\textbf{G}^T$.
%
%\vspace{1.5in}
%
%8) Explain why the row-space of \textbf{G} and the column-space of $\textbf{G}^T$ are equivalent.
%
%\vspace{1in}
%
%9) Find the null-space of $\textbf{G}^{T}$. 
%
%\vspace{1.5in}
%
%This is called the \textit{left nullspace} because it is equivalent to finding the set of vectors $\vec{x}$ which satisfy $\vec{x}^{T} \textbf{A} = \vec{0}$. Notice we have moved $\vec{x}$ to the left-side of \textbf{A}. 
%
%\vspace{0.2in}
%
%10a) What is the dimension of the row-space and left-nullspace of matrix \textbf{G}?
%
%\vspace{0.75in}
%
%10b) How do the dimensions of \textbf{G} relate to the sum: dim( Row-Spc ) + dim( Left-Nullspc) ? 
%
%\vspace{1in}
%
%\section*{Fundamental Spaces}
%11) Detail the four spaces you found above for \textbf{G}. Give their name, dimension and basis. (A 4 x 3 table with column labels would be perfect).
%
%\vspace{3in}
%
%You have now found the four fundamental spaces of the matrix \textbf{G}! These four spaces can be found for any matrix and are used in a variety of operations. We'll look into this a little next week. 
\end{flushleft}
\end{document}