\documentclass{article}
\pagestyle{empty}
\usepackage{amsmath,amssymb,amsfonts}
\usepackage{graphicx}
\usepackage{multicol}
\setlength{\oddsidemargin}{0in} \setlength{\evensidemargin}{0in}
\setlength{\topmargin}{0in} \setlength{\textheight}{8.5in}
\setlength{\textwidth}{6.5in}

\makeatletter
\renewcommand*\env@matrix[1][*\c@MaxMatrixCols c]{%
	\hskip -\arraycolsep
	\let\@ifnextchar\new@ifnextchar
	\array{#1}}
\makeatother

\begin{document}
\begin{flushleft}
	\bfseries{MATH 260, Linear Algebra, Spring `14}\\
	\bfseries{Activity 10:  MORE Eigenvalues and Eigenvectors}\\
	\bfseries{Honor Code:} \hspace{3.5in}\bfseries{Names:}\\
\end{flushleft}
\begin{flushleft}
\vspace{.75in}
Directions:  Everyone should work on the assignment and should fill out their paper.  You are expected to make corrections based on what is presented on the board.  \\
\large The worksheet is due by the end of the day (5pm). \normalsize \\ 
If you need more explanations after class, you can read Section 5.3 of your textbook.\\
\vspace{0.1in}
\Large
In-Class Learning Goals:\\
\normalsize
\begin{enumerate}
\item Be able find eigenspaces
\item Be able to find eigenvectors for repeated eigenvalues
\item Understand how repeated eigenvalues and their eigenvectors affect dimension
%\item Be able to determine if a set of vectors is a basis for a vector space
%\item (stretch) Be able to find the dimension of a set or space (by the end of homework for sure!)

\end{enumerate}

\vspace{0.1in}

\section*{Warm-up:  Subspaces}
Recall from 3 weeks ago (after spring break).

\vspace{0.1in}

\textbf{Vector Subspace Theorem:} a nonempty subset $W$ of a vector space $V$ is a \textbf{subspace} of $V$ if it is closed under addition and scalar multiplication:

\vspace{0.2in}

(i) If $\vec{x},\vec{y}\in W$, then $\vec{x}+\vec{y}\in W$ \hspace{0.5in} (ii) If $\vec{x}\in W$ and $c\in \mathbb{R}$, then $c\vec{x} \in W$

\vspace{0.2in}

1. Sketch the lines: $y=x$ and $ y=x+1$.  One line is a subspace of $\mathbb{R}^2$, one is not. Decide, then explain your reasoning (on paper and to your table).

\vspace{1.5in}

\newpage

\section*{Eigenspaces}

2. a) Last week you had:
$\textbf{A}=
\begin{bmatrix} 
1 & 1 \\ 4&1 
\end{bmatrix}$
 which had eigenvectors:
$\vec{v}_1=\begin{bmatrix} 1 \\ 2 \end{bmatrix}$,
and $\vec{v}_2=\begin{bmatrix} 1 \\ -2 \end{bmatrix}$. \\
\hspace{0.12in} You also found that $\textbf{A}\vec{v}_1 = \begin{bmatrix} 3 \\ 6 \end{bmatrix}$. Sketch $\textbf{A}\vec{v}_1$ and $\vec{v}_1$ (on the same graph).

\vspace{2in}

\hspace{0.12in} b) Looking back at (1), what can you say about $span\{ \vec{v}_1 \}$? 

\vspace{1in}

\hspace{0.12in} c) Sketch $span\{ \vec{v}_2 \}$. Is $span\{ \vec{v}_2 \}$ a subspace of $\mathbb{R}^2$ ?   

\vspace{2in}

\large
We call the $span\{ \vec{v}_i \}$ the \textbf{eigenspace} $\mathbb{E}_{\lambda}$. \\
\normalsize

\vspace{0.2in}

3. What is the dimension of each eigenspace?

\newpage

\section*{Repeated Eigenvalues}

4. What are the eigenvalues and eigenvectors of: 
$\textbf{H}=
\begin{bmatrix}
3 & 0 \\
5 & 3
\end{bmatrix}$

\vspace{3in}

Repeated eigenvalues sometimes have more than one eigenvector though.

\vspace{0.1in}

5. The eigenvalues for this matrix are $\lambda_{1,2}=1$ and $\lambda_3 = 2$. Find the eigenvectors associated with these eigenvalues for the matrix: 
$\textbf{R}=
\begin{bmatrix}
0 & 0 & 2 \\
-1 & 1 & 2 \\
-1 & 0 & 3
\end{bmatrix} $\\


\newpage

We know that eigenvectors always satisfy the 2nd property of vector subspaces (you should be able to explain why!). We also found above that the eigenspace from a single eigenvector is a valid subspace. However, notice that we wrote our eigenspace as $\mathbb{E}_{\lambda}$. But there are TWO eigenvectors with our double eigenvalue. 

\vspace{0.2in}

For notation, lets call the double value, and their vectors $\lambda_1 , \lambda_2$ and $\vec{v}_1 , \vec{v}_2$ respectively. 

\vspace{0.1in}

6. Show that a linear combination of $\vec{v}_1$ and $\vec{v}_2$ (e.g. $\vec{v}_1 + \vec{v}_2$ ) is STILL an eigenvector for $\lambda_{1,2}$. In other words it satisfies the eigenvalue equation:\\
\begin{equation*}
\textbf{R}(\vec{v}_1 + \vec{v}_2 ) = \lambda_1 (\vec{v}_1 + \vec{v}_2 )
\end{equation*}

\vspace{1.75in}

7. State all the eigenspaces and their corresponding dimension for \textbf{R}?

\vspace{1in}

8. Give a geometric interpretation of $span\{ \vec{v}_1 ,  \vec{v}_2 \}$ in $\mathbb{R}^3$ (3-d space). \\
\textit{Hint: this is a special shape in $\mathbb{R}^3$}

\vspace{1in}

The algebraic multiplicity of an eigenvalue is the number of times it is a root (e.g. $a$ in $(\lambda -\lambda_i)^a$)\\
The geometric multiplicity of an eigenvalue is the dimension of its corresponding eigenspace.\\
9. a) Give the algebraic and geometric multiplicity for the eigenvalues of \textbf{R}.

\vspace{0.75in}

\hspace{0.12in} b) Give the algebraic and geometric multiplicity for the eigenvalues of \textbf{H}.

\vspace{0.75in}

\hspace{0.12in} c) How do the algebraic and geometric multiplicity compare to each other?
\newpage

\section*{Distinct Eigenvalues}

There is very useful theorem about distinct eigenvalues:\\
\textbf{Distinct Eigenvalue Theorem:}
Let \textbf{A} be an $n \times n$ matrix. If $\lambda_1 , \lambda_2, \ldots , \lambda_p$ are distinct eigenvalues with corresponding eigenvectors $\vec{v}_1,\vec{v}_2, \ldots, \vec{v}_p$ then $\{ \vec{v}_1,\vec{v}_2, \ldots, \vec{v}_p \}$ is a set of linearly independent vectors. \\

\vspace{0.1in}

10. a) Find the eigenvalues and an eigenvectors for each value for: 
$\textbf{W}=
\begin{bmatrix}
3 & 0 \\
5 & 2 
\end{bmatrix}$

\vspace{2in}

\hspace{0.12in} b) Show that the eigenvectors are linearly independent.

\vspace{1.5in}

11. There's something to be careful of with this theorem though... You found the eigenvectors for \textbf{R} in (3). Are the eigenvectors of \textbf{R} linearly independent?

\vspace{1.5in}

12. If the eigenvectors are linearly independent are you guaranteed to have distinct eigenvalues? 


\end{flushleft}
\end{document}