\documentclass{article}
\pagestyle{empty}
\usepackage{amsmath,amssymb,amsfonts}
\usepackage{graphicx}
\usepackage{multicol}
\setlength{\oddsidemargin}{0in} \setlength{\evensidemargin}{0in}
\setlength{\topmargin}{0in} \setlength{\textheight}{8.5in}
\setlength{\textwidth}{6.5in}

\makeatletter
\renewcommand*\env@matrix[1][*\c@MaxMatrixCols c]{%
	\hskip -\arraycolsep
	\let\@ifnextchar\new@ifnextchar
	\array{#1}}
\makeatother

\begin{document}
\begin{flushleft}
	\bfseries{MATH 260, Linear Algebra, Spring `14}\\
	\bfseries{Activity 6:  Determinants and Cramer's Rule}\\
	\bfseries{Honor Code:} \hspace{3.5in}\bfseries{Names:}\\
\end{flushleft}
\begin{flushleft}
\vspace{.75in}
Directions:  Everyone should work on the assignment and should fill out their paper.  You are expected to make corrections based on what is presented on the board.  \\
\large The worksheet is due by the end of the day (5pm). \normalsize \\ 
If you need more explanations after class, you can read Chapter 3.4 of your textbook.\\
\vspace{0.1in}
\Large
In-Class Learning Goals:\\
\normalsize
\begin{enumerate}
\item Be able to find the determinant of a matrix (both $2\times 2$ AND nXn).
\item Be able to find the `minor' and `cofactor' of elements in a matrix. 
\item Be able to find solutions to a system using \textit{Cramer's Rule}.
\item (Stretch) Discover several properties of determinants including if a matrix has an inverse.
\end{enumerate}

\vspace{0.1in}
\begin{center}
\section*{Warmup: Determinant of a $2 \times 2$}
$|\textbf{A}| = \begin{vmatrix}[rr] a_{11} & a_{12} \\ a_{21} & a_{22} \end{vmatrix} = a_{11} a_{22} - a_{12} a_{21}$\\
\vspace{0.2in}
\textit{Note:} You will need to have this formula memorized!\\
\end{center}
Find the determinant of the following matrix:\\
\begin{center}
$\begin{bmatrix}[rr]
-3 & 1   \\
4 & 2   
\end{bmatrix}
$
\hspace{1in}
$\begin{bmatrix}[rrr]
3 & 0 & -1\\
0 & 0 &  3 \\
0 & 1 & 0
\end{bmatrix}
$
\end{center}
\vspace{2in}

\newpage
\begin{center}
\section{Some Properties and Practice}
\end{center}
For \textbf{Activity 5} you found several inverses and transposes of matrices. Lets use them...\\
\begin{center}
$\textbf{H}=\begin{bmatrix}[rr]
1 & 1\\
4  & 1
\end{bmatrix}$
\hspace{0.5in}
$\textbf{H}^{-1}=\left[
\begin{array}{cc}
- \frac{1}{3} & \frac{1}{3} \vspace{0.1in}\\

\frac{4}{3} & - \frac{1}{3}
\end{array} \right]$
\hspace{0.5in}
$
\textbf{H}^{T}=\begin{bmatrix}[rr]
1 & 4\\
1  & 1
\end{bmatrix}$
\hspace{0.5in}
$\textbf{G}=
\begin{bmatrix}[rrr]
3 & 0 & 3\\
-1 & 2 & 1\\
1 & 1 & 2
\end{bmatrix}$\\
\end{center}

\vspace{0.1in}
a) Find $|\textbf{H}|$.\\
\vspace{0.8in}
b) Find $|\textbf{H}^{-1}|$.\\
\vspace{0.8in}
c) Find $|\textbf{H}^{T}|$.\\
\vspace{0.8in}
d) Find $|\textbf{G}|$.\\
\vspace{1in}
e) Are any of the numbers above the same? Closely related? Try to write each as a generalized property (check these with me).
\newpage
\begin{center}
\section{Finding Minors (and some $2 \times 2$ practice)}
\small This method is only a little harder than visiting an elementary school... \normalsize
\end{center}
\textbf{Definition:} For the $n \times n$ matrix \textbf{A}, the \textit{minor $M_{ij}$} of $a_{ij}$ is an $(n-1) \times (n-1)$ matrix obtained by deleting the $i$th row and the $j$th column of \textbf{A}. \\
\begin{center}
$\textbf{A}=
\begin{bmatrix}
\bullet &\bullet &\bullet & \bullet \\
\bullet &\bullet &\bullet & \bullet \\
\bullet &\bullet &\bullet & \bullet \\
\bullet &\bullet &\bullet & \bullet \\
\end{bmatrix}
=
\begin{bmatrix}[rrrr]
1&2&3&1\\
5&0&1&-2\\
4&0&1&0\\
2&0&3&1
\end{bmatrix}
$
\hspace{0.4in}
$\textbf{M}_{12}=
\begin{bmatrix}
\circ & \circ & \circ & \circ \\
\bullet & \circ & \bullet & \bullet \\
\bullet & \circ & \bullet & \bullet \\
\bullet & \circ & \bullet & \bullet \\
\end{bmatrix}
\rightarrow
\begin{bmatrix}
\bullet & \bullet & \bullet \\
\bullet & \bullet & \bullet \\
\bullet & \bullet & \bullet \\
\end{bmatrix}
=
\begin{bmatrix}[rrr]
5 & 1&-2\\
4 & 1& 0\\
2&3&1
\end{bmatrix}$
\end{center}
a) Find the minor $\textbf{M}_{32}$ of \textbf{A}. To show your work, re-write \textbf{A} first...\\
\vspace{1.5in}
b) Lets find some minors of minors... \\
(i) Find $\textbf{M}_{21}$ of $\textbf{M}_{12}$\\
\vspace{0.75in}
(ii) Find $\textbf{M}_{22}$ of $\textbf{M}_{12}$\\
\vspace{0.75in}
(iii) Find $\textbf{M}_{23}$ of $\textbf{M}_{12}$\\
\vspace{0.75in}
c) Find the determinant of...\\
(i) $\textbf{M}_{21}$\\
\vspace{0.75in}
(ii) $\textbf{M}_{22}$\\
\vspace{0.75in}
(iii) $\textbf{M}_{23}$

\newpage
\begin{center}
\section{Co-Factors}
\end{center}
\textbf{Definition:} The \textit{co-factor} of an element $a_{ij}$ is the scalar: $C_{ij}=(-1)^{i+j}|\textbf{M}_{ij}|$\\
\vspace{0.1in}
Notice that the co-factor is just the determinant of a minor, times $\pm 1$. We'll use this in the next section...
a) Find $C_{21}$ of $\textbf{M}_{12}$\\
\vspace{0.75in}
b) Find $C_{22}$ of $\textbf{M}_{12}$\\
\vspace{0.75in}
c) Find $C_{23}$ of $\textbf{M}_{12}$\\
\vspace{0.75in}
d) Can you find $C_{12}$ of \textbf{A} (based on what you know now)? Why or why not?\\
\vspace{1in}

\begin{center}
\section{Determinants of N x N matrices}
\end{center}
Lets start by finding the determinant of the 3x3 matrix $\textbf{M}_{12}$. \\
To make our notation simpler, lets call $\textbf{M}_{12}$ matrix \textbf{D}.\\
a) Find the following sum: \hspace{0.2in} $d_{21}*C_{21} + d_{22}*C_{22} + d_{23}*C_{23}$ \\
\textit{Hint: You should get a single, scalar number.}\\
\vspace{1in}
b) Try writing the sum for (a) in summation notation (i.e. using: $\Sigma$).\\

\newpage
\LARGE \begin{center} Determinants of N x N matrices (cont.) \end{center} \normalsize
Finding the determinant of an NxN matrix is a recursive process. The definition is:\\
\vspace{0.1in}
\textbf{Definition:} Choose a row $i$ or column $j$, then $|A|=\sum\limits^{n}_{(j \text{ or } i)} a_{ij} C_{ij} = \sum\limits^{n}_{j \text{ or } i} a_{ij} (-1)^{i+j} |\textbf{M}_{ij}| $\\
\vspace{0.1in}
Notice that $\textbf{M}_{ij}$ may not be a $2 \times 2$!! You might have to use this process several times. \\
\vspace{0.1in}
c) Using this definition find the determinant of our original \textbf{A} matrix. \\
\textit{Hint: You should only need to do a few new multiplications and additions, you calculated most of it already.}\\
\vspace{1in}

\end{flushleft}
\end{document}