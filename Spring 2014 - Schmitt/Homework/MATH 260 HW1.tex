\documentclass{article}
\usepackage{amsmath,amssymb,amsfonts}
\usepackage{graphicx}
\usepackage{multicol}
\setlength{\oddsidemargin}{0in} \setlength{\evensidemargin}{0in}
\setlength{\topmargin}{0in} \setlength{\textheight}{8.5in}
\setlength{\textwidth}{6.5in}
\pagestyle{empty}

\begin{document}
\begin{flushleft}
	\bfseries{MATH 260, Homework 1, Spring `14}\\
	\bfseries{Due: January 17, 2014}\\
	\bfseries{Honor Code:} \hspace{3.5in}\bfseries{Name:}\\
	%\hspace{4.37in}\bfseries{Section:}
\end{flushleft}
\begin{flushleft}
\vspace{.25in}

1) How many solutions does each of the following linear systems have?  What are they?

\vspace{0.2in}

a) (10 pts)
\begin{equation*}
\begin{array}{ccccccr}
x  & + &  y & + &  z & = & 6\\
x  & - &  y & + &  z & = & 4\\
3x & - & 2y & + & 3z & = & 13\\
\end{array}
\end{equation*}

\vspace{2.5in}

b) (10 pts)
\begin{equation*}
\begin{array}{ccccccr}
x  & + &  y & + &  z & = & 6\\
x  & - &  y & + &  z & = & 4\\
2x & + &  y & + & 3z & = & 13\\
\end{array}
\end{equation*}

\vspace{2.5in}

2) (5 pts) Given two linear equations in three variables, what are the possibilities for the numbers of solutions (such as ``there could be two or five solutions")?




\end{flushleft}

\end{document}
