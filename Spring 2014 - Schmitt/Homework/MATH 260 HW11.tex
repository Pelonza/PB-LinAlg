\documentclass{article}
\usepackage{amsmath,amssymb,amsfonts}
\usepackage{graphicx}
\usepackage{multicol}
\setlength{\oddsidemargin}{0in} \setlength{\evensidemargin}{0in}
\setlength{\topmargin}{0in} \setlength{\textheight}{8.5in}
\setlength{\textwidth}{6.5in}
\pagestyle{empty}

\begin{document}
\begin{flushleft}
	\bfseries{MATH 260, Homework 11, Spring `14}\\
	\bfseries{Due: April 21, 2014}\\
	\bfseries{Honor Code:} \hspace{3.5in}\bfseries{Name:}\\
	%\hspace{4.37in}\bfseries{Section:}
\end{flushleft}
\begin{flushleft}
\vspace{.25in}

As in Activity 11, we will use the matrix: $\textbf{A} = \begin{bmatrix} 1 & 1 \\ 4 & 1 \end{bmatrix}$ with eigenvalues $\lambda_1 = 3$ and $\lambda_2 = -1$ with corresponding eigenvectors $\vec{v}_1 = \begin{bmatrix} 1 \\ 2 \end{bmatrix}$ and $\vec{v}_2 = \begin{bmatrix} 1 \\ -2 \end{bmatrix}$ respectively.

\vspace{0.2in}

Powering a diagonal matrix is easy: $\textbf{D}^k = \begin{bmatrix}
d_1^k & 0 & \ldots & 0 \\
0 & d_2^k & \ldots & 0 \\
\vdots & \vdots & \ddots & \vdots \\
0 & 0 & \ldots & d_n^k
\end{bmatrix}$.  However, powering any non-diagonal square matrix isn't quite as easy.  Let's see if we can abuse the diagonalizability of $\textbf{A}$ to make it easier, though.

\vspace{0.2in}

1) (8 pts) Compute $\textbf{PD}^2 \textbf{P}^{-1}$ and $\textbf{A}^2 $.  What do you notice?

\vspace{3in}

2) (8 pts) Compute $\textbf{PD}^3 \textbf{P}^{-1}$.  Is it equal to $\textbf{A}^3 = \begin{bmatrix} 13 & 7 \\ 28 & 13 \end{bmatrix}$?

\vspace{3in}

3) (14 pts) Give a general equation to compute $\textbf{A}^k$.  This works for every diagonalizable matrix $\textbf{A}$.  Why does this work?  Recall that $\textbf{A} = \textbf{PDP}^{-1}$.

\end{flushleft}
\end{document}
